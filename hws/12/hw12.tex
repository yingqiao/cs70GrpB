\documentclass[]{article}
% packages
\usepackage{../../cs70}
\usepackage{../../markup}
\usepackage{enumerate}
\usepackage{hyperref}
%% \usepackage{framed}
%% \usepackage{MnSymbol}
%% \usepackage{epstopdf}
\usepackage{color}
%% \usepackage[]{amsmath}
%% \usepackage{graphicx}
%% \usepackage{amssymb}
%% \usepackage{parskip}
%% \usepackage{rotating}
%% \usepackage{float}
%% \usepackage{multirow}
%% \usepackage{subcaption}
%% \usepackage{indentfirst}
%% \usepackage[left=1.5in, right=1.0in, top=1.0in, bottom=1.0in]{geometry}

\newtheorem{Definition}{Definition}
\newtheorem{Lemma}{Lemma}
\newtheorem{Proof}{Proof}
\newtheorem{Theorem}{Theorem}

\newif\ifsolutions
\newif\ifmotivation
\motivationtrue
\motivationfalse
\solutionstrue
%\solutionsfalse %flag for solutions

\renewcommand{\answer}[1]{{\color{mydarkblue}\textbf{Solution:}#1}}
\definecolor{mydarkblue}{rgb}{0,0.25,1}

\def \exx {\mathbb{E}}
\def\title{Homework 12}

\begin{document}

\maketitle
\config{hwnum}{12}
\config{homework-due}{04/21/2014 13:00}
\config{grades-due}{04/28/2014 13:00}
\vspace{0.5em}
{\Large{\textbf{This homework is due April 21 2014, at 12:00 noon.}}}

\begin{qunlist}

\qns{Independent Random Variables}\\
Find four random variables taking values in $\{-1,1\}$ so that any three are independent but all four are not.

\ifsolutions{ \answer{
Let $X_1,X_2,X_3,X_4$ be i.i.d random variables with $P(X_i=1)=P(X_i=-1)=1/2$. Let $X_4=X_1X_2X_3$. Check that $X_1,X_2,X_3,X_4$ are four random variables such that any three are independent but all four are not. For example, they are not all independent because
\[
P(X_1=1,X_2=1,X_3=1,X_4=1)=1/8 \neq
P(X_1=1)P(X_2=1)P(X_3=1)P(X_4=1)
\]
}}
\fi

\qns{Expectation Basics} \\ For discrete random variables $X$ and $Y$:

\begin{enumerate}[a)]
\qpart
\item Show that for constants $c$ and $d$, $\mathbb{E}[\min(X,Y)] + \mathbb{E}[\max(X,Y)] = \mathbb{E}[X] + \mathbb{E}[Y]$.



%-show that if $X$ and $Y$ are independent discrete random variables, then \[\mathbb{E}[XY] = \mathbb{E}[X] \mathbb{E}[Y].\]  Does this property imply independence?

\qpart
\item Show that $\mathbb{E}[cX + dY] = c\mathbb{E}[X] + d \mathbb{E}[Y]$.

\ifsolutions{ \answer{ Beginning with the LHS:
		\begin{align*}
		\mathbb{E}[cX + dY] & = \sum_{x,y} (cx+dy) \mathbb{P}(x,y) \\
		& = c\sum_{x,y} x \mathbb{P}(x,y) + d\sum_{x,y} y \mathbb{P}(x,y) \\
		& = c \sum_x x \sum_y \mathbb{P}(x,y) + d \sum_y y \sum_x \mathbb{P}(x,y) \\
		& = c\sum_x x \mathbb{P}(x) + d\sum_y y \mathbb{P}(y) \\
		& = c\mathbb{E}[X] + d \mathbb{E}[Y].
		\end{align*}
}}
\fi

\end{enumerate}

\qns{Variance of Independent R.V.s}

In this question, we will consider the variance of a sum of random variables. We already know how to compute the expectation of a sum of random variables because Expectation is linear ($\mathbb{E}[X+Y] = \mathbb{E}[X] + \mathbb{E}[Y]$).
\begin{enumerate}
\qpart
\item[a)] We want to work with random variables that are pair-wise independent. This means that for a set of random variables: $X_1, X_2, \cdots, X_n$, $\forall{i \neq} j $ Pr$[X_i \cap X_j] = Pr[X_i]Pr[X_j]$. We want to come up with a set of events that are pair-wise independent, even if they may not be totally independent. Let us define for $1 \leq i \leq n/2$ even, $X_i$ represents the event that the $i$ time we flip a fair coin is a head. Let us define for $n/2 + 1 \leq i \leq n$, $X_i$ represents the event that $X_{i-n/2} = X_{i-1}$. Argue that these events are all pair-wise independent, but give a set of three events that are not independent (that is give an $X_i$, $X_j$, and $X_k$ such that $Pr[X_i \cap X_j \cap X_k] \neq Pr[X_i]Pr[X_j]Pr[X_k]$)
\qpart
\item[b)] For the rest of the problem, we consider a set of events $X_1, X_2, \cdots, X_n$ such that they are pair-wise independent (we are no longer working with the particular example in part a). In order to compute the variance of the sum of these variables we wish to prove the following lemma: \\ \textbf{Lemma:} If $A$ and $B$ are pair-wise independent random variables, then $\exx[AB] = \exx[A] \exx[B]$ \\ Prove this lemma. (Hint: Let $\Aset$ be the set of all possible values of $A$ and let $\Bset$ be the set of all possible values of $B$. Now use the definition of expectation, $\exx[AB] = \sum\limits_{a \in \Aset} \sum\limits_{b \in \Bset} ab \cdot Pr[A=a \cap B = b]$).
\qpart
\item[c)] Let $X$ be a sum of pair-wise independent random variables. So $X = \sum\nolimits_{i=1}^n X_i$. We wish to solve for $Var[X]$. We break this into a few steps. First note that $Var[X] = \exx[X^2] - \exx[X]^2$. \\ Prove that $\exx[X^2] = \sum\nolimits_{i=1}^n \exx[X_i^2] + \sum\limits_{i \neq j} \exx[X_iX_j]$ \\ (Hint: Note that $X = X_1 + X_2 + \cdots X_n$. What happens when you expand out $(X_1 + X_2 + \cdots X_n)^2$?)
\qpart
\item[d)] Prove that $\exx[X]^2 = \sum\nolimits_{i=1}^n \exx[X_i]^2 + \sum\limits_{i \neq j} \exx[X_i] \exx[X_j]$
\qpart
\item[e)] We combine c and d to show $Var[X] = \sum\nolimits_{i=1}^n \exx[X_i^2] + \sum\limits_{i \neq j} \exx[X_iX_j] -  \sum\nolimits_{i=1}^n \exx[X_i]^2 - \sum\limits_{i \neq j} \exx[X_i] \exx[X_j]$. Now use the lemma and rearrange to show that $Var[X] = \sum\nolimits_{i=1}^n \exx[X_i^2] - \exx[X_i]^2$
\qpart
\item[f)] Finally reason that $Var[X] = Var[\sum\nolimits_{i=1}^n X_i] =    \sum\nolimits_{i=1}^n Var[X_i]$. This is true if for all $i \neq j$, $X_i$ and $X_j$ are pair-wise independent. Does this equation hold of $X_i$ and $X_j$ are not pair-wise independent?

\end{enumerate}
  
\qns{Binomial Distribution}
In this question we will derive the Binomial Distribution. The \underline{distribution} of a discrete random variable~$X$ is the collection of values $\{(a \Pr[X=a]):a\in\Aset\}$, where $\Aset$ is the set of all possible values taken by~$X$. The Binomial Distribution is used when we have an event that always has a probability $p$ of occurring during each trial, and we run $n$ trials. If this sounds familiar, it's been the focus of almost all of our labs! For the target of these problems, we imagine we are tossing a weighted-coin with probability $p$ of heads. Remember that $0 < p < 1$.
\begin{enumerate}
\qpart
\item[(a)] What is the probability that after $n$ trials, we see exactly $n$ heads? Recall that each coin flip has probability $p$ of being a head.
\qpart
\item[(b)] What is the probability that after $n$ trials, we see exactly $0$ heads?
\qpart
\item[(c)] How many sequences of $n$ coin flips have exactly 1 head?
\qpart
\item[(d)] How many sequences of $n$ coin flips have exactly $k$ heads? Let  $0 \leq k \leq n$ 
\qpart
\item[(e)] What is the probability of a particular outcome of $k$ heads occurring? For example, what is the probability the first $k$ trials are heads, and the rest of the trials are tails? Let  $0 \leq k \leq n$ 
\qpart
\item[(f)] Combine part d and e to compute the probability that we flip $k$ heads in $n$ trials. Call this Pr[$X = k$] since our random variable $X$ counts the number of heads after $n$ flips. Let  $0 \leq k \leq n$ 
\qpart
\item[(g)] We have now specified the distribution of $X$ where $X$ counts the number of heads after $n$ flips of a biased coin with a probability $p$ of flipping a head. In order to prove that this is indeed a distribution, we must verify that $\sum\nolimits_{i=1}^n Pr[X=i] = 1$ Verify that your distribution is valid (all the probabilities sum to 1) when $n=2$ and $p=1/3$.
\qpart
\item[(h)] Now we want to prove mathematically that this holds for all values of $n$ and $p$. The Binomial Theorem states that $(a+b)^n = \sum\nolimits_{k=0}^n \binom{n}{k} a^{n-k}b^k $. We will use the Binomial Theorem in our proof, so give a combinatorial argument to support the Binomial Theorem. Why is the coefficient of each term $\binom{n}{k}$? It may help to try some examples for $n=3$ and $n=4$.
\qpart
\item[(i)] Now plug in the correct values for $a$ and $b$ so that this looks like your equation of $\sum\nolimits_{i=1}^n Pr[X=i]$. You should be able to prove that the sum of these probabilities is indeed 1.
\qpart
\item[(j)] We can say $X \sim B(n,p)$ Which is read as "$X$ is a binomial random variable with parameters $n$ and $p$. We lastly want to calculate $\mathbb{E}[X]$. An easy way to do this is to see $X$ as as sum of independent random variables. Each $X_i$ is an indicator variable that is $1$ if the $i^{\text{th}}$ flip is heads and $0$ if the $i^{\text{th}}$ flip is tails. Argue briefly that $X = \sum\nolimits_{i=1}^nX_i$.
\qpart
\item[(k)] Use the linearity of expectation to calculate $\mathbb{E}[X]$.
\qpart
\item[(l)] Looking at what this expectation is, remark on one empirical finding from a virtual lab or the lab we did during discussion that is explained by the expectation you just calculated. Congratulate for finishing a problem by GSI lots-of-parts 
\end{enumerate}


%\begin{enumerate}[a)]
%\qpart 
%\item 
%\end{enumerate}

\qns{Winning the Lottery} \\ Suppose that every day, Lily buys a lottery ticket, and will only stop buying lottery tickets when she wins.  Let $p$ be the probability that on any given day, Lily wins the lottery.  Let $X$ represent the totaly number of lottery tickets Lily buys.

\begin{enumerate}[a)]

\qpart
\item What is the probability that Lily only buys 1 lottery ticket (i.e. $X=1$)?  What is the probability $X=2$?  What is the probability $X=x$?  We can denote this $f(x)$.
%This is the probability mass function for $X$, which we can denote $f(x)$.
Show that $f(x)$ is a proper probability distribution.

\ifsolutions{ \answer{ The probability of getting $(x-1)$ non-winning lottery tickets is $(1-p)^{x-1}$. Therefore the probability of getting $(x-1)$ non-winning tickets, then a winning ticket is
	\[ P(X=x) = (1-p)^{x-1} p. \]
	To show this is a valid probability distribution, we need to make sure the sum of $f(x)$ over all possible $x$ is 1. Using the formula for an infinite geometric series,
	\[ \sum_{x=1}^{\infty} f(x) = p \sum_{x=1}^{\infty} (1-p)^{x-1} = p \cdot \frac{1}{1-(1-p)} = 1. \]
}}\fi

\qpart
\item (Lab) Using a computer, simulate $m=1,000$ trials of Lily buying lottery tickets until she gets a winner. Use $p=0.2$, and plot a histogram with the number of lottery tickets on the $x$-axis and the counts of each of these outcomes on the $y$-axis.  What is the average number of lottery tickets Lily has to buy?

\qpart
\item (Lab) Now, rescale the histogram so that the sum of all the bars is 1.  Then overlay $f(x)$ from part (a).

\qpart
\item Using your formula for $f(x)$, compute the $\mathbb{E}(X)$, the expected value of $X$. How does this relate to the average number of lottery tickets from part (b)? %What is the expected value of $X$?

%\ifsolutions{ \answer{ 
%\[ \sum_{x=0}^{\infty} x f(x) = p\sum_{x=0}^{\infty} x(1-p)^x \]
%}}\fi

\qpart
\item Compute the cumulative mass function (cmf) for $X$, $F(x) = \mathbb{P}(X \leq x)$.

\ifsolutions{ \answer{ 
	We can use the formula for a finite geometric series to find $F(x)$,
	\[ F(x) = p \sum_{y=1}^x (1-p)^{y-1} = p \cdot \frac{1 - (1-p)^x}{1 - (1-p)} = 1 - (1-p)^x. \]
}}\fi

\qpart
\item (Optional Lab) Use your histogram from part (c) to plot an ``empirical cmf'' (i.e. for each number of tickets $x$, plot the fraction of trials where Lily bought $x$ or fewer tickets). Then overlay $F(x)$.

\end{enumerate}





\qns{007} \\
James Bond is imprisoned in a cell from which there are three possible ways to escape: an air-conditioning
duct, a sewer pipe and the door (which is unlocked). The air-conditioning duct leads him on a two-hour trip
whereupon he falls through a trap door onto his head, much to the amusement of his captors. The sewer pipe
is similar but takes five hours to traverse. Each fall produces temporary amnesia and he is returned to the
cell immediately after each fall. Assume that he always immediately chooses one of the three exits from the
cell with probability 1 . On the average, how long does it take before he realizes that the door is unlocked 3
and escapes?
[HINT: This problem may be tricky unless you think about it the right way. Consider the outcome of Bond's
first attempt, and see how the expected time to escape changes as a result of this. In particular, you should
be able to prove that $E(T) = \frac{1}{3}E(T|A) + \frac{1}{3}E(T|S) + \frac{1}{3}E(T|D)$, where the random variable T is the time 333
to escape, and E(T |A) means the expected time to escape given that Bond went through the AC-duct in his first attempt, etc. To solve the problem, think how $E(T |A)$, $E(T |S)$ and $E(T |D)$ relate to $E(T )$.]

\ifsolutions{ \answer{
Using the notation in the extremely helpful hint, $E[T]=E[T\,|\, A]\Pr[A]+E[T\,|\, S]\Pr[S]+E[T\,|\, D]\Pr[D]=\frac{1}{3}(E[T\,|\, A]+E[T\,|\, S]+E[T\,|\, D])$.
Note that $E[T\,|\, A]=E[T]+2$ by the memorylessness of the situation.
Similarly, $E[T\,|\, S]=E[T]+5$. Lastly, $E[T\,|\, D]=0$ because
Mr. Bond escapes immediately. So we have that $E[T]=\frac{1}{3}(E[T]+2+E[T]+5+0)$.
Solving this for $E[T]$, we get that $E[T]=\fbox{\ensuremath{\mathbf{007}}}$.
}}
\fi

\qns{Balls and boxes}\\
Suppose $\alpha n$ balls are dropped at random into $n$ boxes so that all $n^{\alpha n}$ assignments have equal probability. Let $N$ be the number of empty boxes.
\begin{itemize}
\item[a)] Compute $E(N)$
\item[b)] If $X=N/n$ is the proportion of empty boxes, what are the limiting values of $E(X)$ as $n \rightarrow \infty$?
\item[c)] In a real situation, given the value of $X$, how would you estimate $\alpha$ if it is unknown?
\end{itemize}



\ifsolutions{ \answer{
\begin{itemize}
\item[a)] $Y_i=1$ if box $i$ is empty, 0 otherwise;
\[
E(N)=E(\sum_{i=1}^n Y_i)=\sum_{i=1}^n E(Y_i)=\sum_{i=1}^n P(Y_i=1)
=\sum_{i=1}^n (\frac{(n-1)^{\alpha n}}{n^{\alpha n}}) = n(1-\frac{1}{n})^{\alpha n}
\]
\item[b)] Using the fact that 
\[
\lim_{n\rightarrow \infty}(1-\frac{1}{n})^{\alpha n} = e^{\alpha}
\]

\[
E(X)=E(\frac{N}{n})=\frac{E(N)}{n}=(1-\frac{1}{n})^{\alpha n}
\]

\[\lim_{n \rightarrow \infty}E(X) = e^{\alpha} \]

\item[c)]
If $n$ is large and we know $X_n$, then our estimate of $\alpha$ would be $\hat{\alpha}=logX_n$


\end{itemize}
}}
\fi

\qns{Write your own problem} \\
Write your own problem related to this week's material and solve it. You may still work in groups to brainstorm problems, but each student should submit a unique problem. What is the problem? How to formulate it? How to solve it? What is the solution?

\qns{Midterm question 3} \\
Re-do midterm question 3.
\qns{Midterm question 4} \\
Re-do midterm question 4.
\qns{Midterm question 5} \\
Re-do midterm question 5.
\qns{Midterm question 6} \\
Re-do midterm question 6.
\qns{Midterm question 7} \\
Re-do midterm question 7.
\qns{Midterm question 8} \\
Re-do midterm question 8.
\qns{Midterm question 9} \\
Re-do midterm question 9.
\qns{Midterm question 10} \\
Re-do midterm question 10.
\qns{Midterm question 11} \\
Re-do midterm question 11.
\qns{Midterm question 12} \\
Re-do midterm question 12.
\qns{Midterm question 13} \\
Re-do midterm question 13.
\qns{Midterm question 14} \\
Re-do midterm question 14.

\end{qunlist}


%OPTIONAL PROBLEMS
%
%\begin{qunlist}
%
%\qns{Expectation Warm up}
% 
% I think it would be good to give students a warm up question on expectation. We give them a distribution, say $Pr(X = 1) = 1/2, Pr (X=2) = 1/3, Pr (x=3) = 1/6$ Then we ask them to compute the expectation.
% 
% Other warm up problems would be given the expectation, asking some questions about the probability of X. (for example, $X\geq 0, Pr(X=0) = 1/2, \mathbb{E}[X] = 10.$ True or false $Pr(X \geq 20) > 0$
%
%
%\qns{Expected number of fixed points in a permutation}
%
%It's 1, done easily through linearity of expectation
%
%
%\qns{Bubblesort} \\
%The well-known Bubblesort algorithm sorts a list $a_1, a_2, \ldots, a_n$ of numbers by repeatedly swapping adjacent numbers that are inverted (i.e., in the wrong relative order) until there are no remaining inversions. Suppose that the input to Bubblesort is a random permutation of the numbers $a_1, a_2, \ldots, a_n$, so that all $n!$ orderings are equally likely, and that all the numbers are distinct. What is the expected number of swaps performed by Bubblesort?
%
%\qns{Machine Failures} \\
%Two faulty machines, $M_1$ and $M_2$, are repeatedly run synchronously in parallel (i.e., both machines execute one run, then both execute a second run, and so on). On each run, $M_1$ fails with probability $p_1$ and $M_2$ with probability $p_2$, all failure events being independent. Let the random variables $X_1$, $X_2$ denote the number of runs until the first failure of $M_1$, $M_2$ respectively; thus $X_1$, $X_2$ have geometric distributions with parameters $p_1$, $p_2$ respectively.
%Let $X$ denote the number of runs until the first failure of either machine. Show that $X$ also has a geometric distribution, with parameter $p_1 + p_2 − p_1p_2.$
%
%\qns{St. Petersburg Paradox}
%
%Consider a single-player game where the pot initially contains \$1. Then a fair coin is tossed until it comes up tails, at which point the player gets all the money in the pot. Each time a head comes up, the amount of money in the pot is doubled.  So if a tails comes up first, the player gets \$1.  If the first tails comes up second, the player gets \$2, if tails comes up third, the player gets \$4, and so on.
%
%\begin{enumerate}[a)]
%
%\qpart
%\item For $m=1,2,3, \cdots 1,000$, simulated $m$ trials of this game.  For each $m$, compute the average winnings of the player.  Plot the average winnings on the $y$-axis vs. $m$ on the $x$-axis.  What do you notice?  Are the average winnings converging to a specific value?
%
%\qpart
%\item Now compute the expected value of the winnings.  How can you reconcile this result with part (a)?
%\end{enumerate}
%

%\qns{Problem from Chung-Wei} \\
%TODO get figs from Chung-Wei if you use this problem
%There are $n$ vertices $\{v_1,v_2,\ldots,v_n\}$ on a plane, and the coordinates of $v_i$ are $(x_i,y_i)$. The bounding box of $v_i$ and $v_j$ is a rectangular $\{(x,y)|(x-x_i)(x-x_j)\leq 0\mbox{ and }(y-y_i)(y-y_j)\leq 0\}$. An edge is added between two vertices if and only if there is no other vertex inside or on the boundary of the bounding box of the two vertices. Prove that the expected number of edges is smaller than $(2n+2)\ln n$.
%
%\ifsolutions{ \answer{
%The number of edges in the OASG is $O(n^2)$ in the worst case, but Theorem~\ref{theorem:edge-random-case} shows that the expected number of edges in the OASG is $O(n\lg n)$. To compute the expected number of edges in the OASG, several notations and lemmas are first given.
%
%\begin{Definition}\label{def:g1}
%Given an instance, $G_1:(V_1,E_1)$ is defined as the OASG.
%\end{Definition}
%
%\begin{Definition}\label{def:g2}
%Given an instance, $G_2:(V_2,E_2)$ is defined as the OASG constructed after regarding all corner-vertices as pin-vertices and removing all obstacles from the plane, \emph{i.e.}, there are totally $n$ pin-vertices and no obstacle on the plane.
%\end{Definition}
%
%\begin{Definition}\label{def:g3}
%Given an instance, $G_3:(V_3,E_3)$ is defined as the OASG constructed after regarding all corner-vertices as pin-vertices, removing all obstacles from the plane, and moving a small enough distance, $\varepsilon$, for vertices with the same $x$-coordinate or $y$-coordinate so that all vertices have different $x$-coordinates and $y$-coordinates.
%\end{Definition}
%
%%\begin{figure}
%%\centering
%%\psfig{figure=nlogn-proof1.eps,width=12cm}
%%\caption{\small An example of $G_1$, $G_2$, and $G_3$. (a) Given an instance, (b) $G_1$ has fewer edges than (c) $G_2$ which has fewer edges than (d) $G_3$.} \label{fig:nlogn-proof1}
%%\end{figure}
%
%See Figure~\ref{fig:nlogn-proof1} for an example of $G_1$, $G_2$, and $G_3$.
%
%\begin{Definition}\label{def:gamma}
%Given an instance, a permutation $\Gamma$ is defined by labelling from 1 to $n$ for the vertices in $G_3$ by the order of their $y$-coordinates and permuting these labelled numbers by the order of their $x$-coordinates.
%\end{Definition}
%
%\begin{Definition}\label{def:n4}
%Given a permutation $\Gamma$, $N_4$ is defined as the number of pairs $(i,j)$, where $1\leq i,j\leq n$, and there is no integer whose value is between $i$ and $j$ and whose position in the permutation $\Gamma$ is between the positions of $i$ and $j$.
%\end{Definition}
%
%%\begin{figure}
%%\centering
%%\psfig{figure=nlogn-proof2.eps,width=12cm}
%%\caption{\small (a) Given the $G_3$, $\Gamma$ is defined by (b) labelling from 1 to $n$ for vertices in $G_3$ by the order of their $y$-coordinates and (c) permuting these labelled numbers by the order of their $x$-coordinates, resulting in $\Gamma=<2,1,3,5,4>$. (d) The six pairs make $N_4=6$ which is equal to the number of edges in $G_3$.} \label{fig:nlogn-proof2}
%%\end{figure}
%
%%See Figure~\ref{fig:nlogn-proof2} for an example of the permutation $\Gamma$ and $N_4$. There are the following lemmas.
%
%\begin{Lemma}\label{lemma:e1-e2}
%$|E_1|\leq |E_2|$.
%\end{Lemma}
%
%\begin{Proof}
%For any edge in $G_1$, it is always in $G_2$, but there are some edges in $G_2$ whose corresponding edges in $G_1$ are blocked by obstacles. As a result, $|E_1|\leq |E_2|$.
%\end{Proof}
%
%\begin{Lemma}\label{lemma:e2-e3}
%$|E_2|\leq |E_3|$.
%\end{Lemma}
%
%\begin{Proof}
%For any edge $(v_1,v_2)$ in $G_2$, there is no other vertex inside or on the boundary of the bounding box of $v_1$ and $v_2$. Because the moving of each vertex is small enough, there is still no other vertex inside or on the boundary of the bounding box of $v_1$ and $v_2$. As a result, $(v_1,v_2)$ is still in $G_3$, and $|E_2|\leq |E_3|$.
%\end{Proof}
%
%\begin{Lemma}\label{lemma:e3-n4}
%$|E_3|=N_4$.
%\end{Lemma}
%
%\begin{Proof}
%For vertices $v_1$, $v_2$, and $v_3$ in $G_3$, the $y$-coordinate of $v_3$ is between those of $v_1$ and $v_2$ if and only if its labelled number is between those of $v_1$ and $v_2$; the $x$-coordinate of $v_3$ is between those of $v_1$ and $v_2$ if and only if its position in the permutation $\Gamma$ is between those of $v_1$ and $v_2$. For any edge $(v_1,v_2)$ in $G_3$, there is no other vertex inside or on the boundary of the bounding box of $v_1$ and $v_2$, resulting in a pair of $(i,j)$ in $\Gamma$ where there is no integer whose value is between $i$ and $j$ and whose position in $\Gamma$ is between those of $i$ and $j$. On the other hand, a pair of $(i,j)$ in $\Gamma$ where there is no integer whose value is between $i$ and $j$ and whose position in $\Gamma$ is between those of $i$ and $j$ means that there is no other vertex inside or on the boundary of the bounding box of the two corresponding vertices. As a result, $|E_3|=N_4$ due to the one-to-one mapping.
%\end{Proof}
%
%\begin{Lemma}\label{lemma:n4}
%The expected value of $N_4$ is $O(n\lg n)$.
%\end{Lemma}
%
%\begin{Proof}
%For any pair $(i,i+j)$ where $i\geq 1$, $j\geq 1$, and $i+j\leq n$, there are $(j+1)!$ permutations from $i$ to $i+j$. Among these $(j+1)!$ permutations, $(i,i+j)$ is counted if and only if $i$ and $i+j$ are permuted successively; otherwise, there is at least an integer between $i$ and $i+j$ whose position in $\Gamma$ is between positions of $i$ and $j$. Because there are $2j!$ permutations from $i$ to $i+j$ where $i$ and $i+j$ are permuted successively, and there are $n!$ permutation from 1 to $n$, the pair $(i,i+j)$ is counted $n!\frac{2j!}{(j+1)!}$ times among all permutations from 1 to $n$.
%
%Because there are $(n-j)$ types of pairs $(i,i+j)$, the total count is $\sum_{j=1}^{n-1}((n-j)n!\frac{2j!}{(j+1)!})$ among all permutations from 1 to $n$. Therefore, the expected value of $N_4$ is
%\begin{eqnarray*}
% & {\textstyle        } & {\textstyle \frac{1}{n!}\sum_{j=1}^{n-1}\left((n-j)n!\frac{2j!}{(j+1)!}\right)}            \nonumber\\
% & {\textstyle { }={ }} & {\textstyle \sum_{j=1}^{n-1}\left((n-j)\frac{2}{j+1}\right)}                               \nonumber\\
%%& {\textstyle { }={ }} & {\textstyle 2\sum_{j=1}^{n-1}\left(\frac{n}{j+1}-\frac{j}{j+1}\right)}                     \nonumber\\
% & {\textstyle { }={ }} & {\textstyle 2n\sum_{j=1}^{n-1}\frac{1}{j+1}-2\sum_{j=1}^{n-1}\left(1-\frac{1}{j+1}\right)} \nonumber\\
% & {\textstyle { }={ }} & {\textstyle (2n+2)\sum_{j=1}^{n-1}\frac{1}{j+1}-2(n-1)}                                    \nonumber\\
% & {\textstyle { }<{ }} & {\textstyle (2n+2)\int_1^n\left(\frac{1}{x}\right)dx-2(n-1)}                             \nonumber\\
% & {\textstyle { }={ }} & {\textstyle (2n+2)\ln n-2(n-1)}                                                           \nonumber
%\end{eqnarray*}
%As a result, the expected value of $N_4$ is $O(n\lg n)$.
%\end{Proof}
%
%\begin{Theorem}\label{theorem:edge-random-case}
%The expected number of edges in the OASG is $O(n\lg n)$.
%\end{Theorem}
%
%\begin{Proof}
%Given an instance, by Lemmas~\ref{lemma:e1-e2}, \ref{lemma:e2-e3}, and \ref{lemma:e3-n4}, the number of edges in the OASG is less than its corresponding $N_4$. By Lemma~\ref{lemma:n4}, the expected value of $N_4$ is $O(n\lg n)$. As a result, the expected number of edges in the OASG is $O(n\lg n)$ since the probability for each kind of the permutation $\Gamma$ is the same.
%\end{Proof}
%}} \fi
%    
%\end{qunlist}

\end{document}
