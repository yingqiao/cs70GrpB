\documentclass[]{article}
% packages
\usepackage{../../cs70}
\usepackage{../../markup}
\usepackage{enumerate}
\usepackage{hyperref}
%% \usepackage{framed}
%% \usepackage{MnSymbol}
%% \usepackage{epstopdf}
\usepackage{color}
%% \usepackage[]{amsmath}
%% \usepackage{graphicx}
%% \usepackage{amssymb}
%% \usepackage{parskip}
%% \usepackage{rotating}
%% \usepackage{float}
%% \usepackage{multirow}
%% \usepackage{subcaption}
%% \usepackage{indentfirst}
%% \usepackage[left=1.5in, right=1.0in, top=1.0in, bottom=1.0in]{geometry}

\newif\ifsolutions
\newif\ifmotivation
\motivationtrue
\motivationfalse
\solutionstrue
%\solutionsfalse %flag for solutions

\renewcommand{\answer}[1]{{\color{mydarkblue}\textbf{Solution:}#1}}
\definecolor{mydarkblue}{rgb}{0,0.25,1}

\def\title{Homework 12}

\begin{document}

\maketitle
\config{hwnum}{12}
\config{homework-due}{04/21/2014 13:00}
\config{grades-due}{04/28/2014 13:00}
\vspace{0.5em}
{\Large{\textbf{This homework is due April 21 2014, at 12:00 noon.}}}

\begin{qunlist}
  
\qns{Binomial variable}

I always thought it would be interesting if instead of simply presenting a distribution to students, we could have them derive it in homework by breaking down the steps at a time. Students have already seen examples when p =1/2 so I figured we could break down the derivation using a dice:
\begin{itemize}
\item Introduce the idea of a weighted coin
\item have them compute probability of a particular sequence with 1 head
\item ask them how many sequences have exactly 1 head
\item combine the two previous expressions to see the probability of one head.
\item do the same two steps for 2 heads.
\item do the same two steps for k heads.
\item perhaps we could walk them through the proof that this is indeed a distribution (sums to 1)
\end{itemize}

%\begin{enumerate}[a)]
%\qpart 
%\item 
%\end{enumerate}

\qns{Expectation Warm up}
 
 I think it would be good to give students a warm up question on expectation. We give them a distribution, say $Pr(X = 1) = 1/2, Pr (X=2) = 1/3, Pr (x=3) = 1/6$ Then we ask them to compute the expectation.
 
 Other warm up problems would be given the expectation, asking some questions about the probability of X. (for example, $X\geq 0, Pr(X=0) = 1/2, \mathbb{E}[X] = 10.$ True or false $Pr(X \geq 20) > 0$

%\begin{enumerate}[a)]
%\qpart 
%\item
%\end{enumerate}

\qns{The James Bond question from previous HWs is really cool expectation}

%\begin{enumerate}[a)]
%\qpart
%\item 
%\end{enumerate}

\qns{Expected number of fixed points in a permutation}

It's 1, done easily through linearity of expectation

%\begin{enumerate}
%\qpart 
%\item[a)] 
%\end{enumerate}


 \qns{Other Expectation Stuff}
 
 show that $\mathbb{E}[min(X,Y)] + \mathbb{E}[max(X,Y)] = \mathbb{E}[X] + \mathbb{E}[Y]$

%\begin{enumerate}[a)]
%\qpart
%\item 
%\end{enumerate}


\qns{Write your own problem} \\
Write your own problem related to this week's material and solve it. You may still work in groups to brainstorm problems, but each student should submit a unique problem. What is the problem? How to formulate it? How to solve it? What is the solution?
    
\end{qunlist}

\end{document}
