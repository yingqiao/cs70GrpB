\documentclass[]{article}
% packages
\usepackage{../../cs70}
\usepackage{../../markup}
\usepackage{enumerate}
\usepackage{hyperref}
\usepackage{color}

\newif\ifsolutions
% \solutionstrue
\solutionsfalse %flag for solutions

\renewcommand{\answer}[1]{{\color{mydarkblue}\textbf{}#1}}
\definecolor{mydarkblue}{rgb}{0,0.25,1}

\def\title{Homework 6}

\begin{document}

\maketitle
\config{hwnum}{5}
\config{homework-due}{03/03/2014 13:00}
\config{grades-due}{03/10/2014 13:00}
\vspace{0.5em}
{\Large{\textbf{This homework is due March 03 2014, at 12:00 noon.}}}

\begin{qunlist}
  
  \qns{Twitch Plays Pokemon ... a little differently}

  \textbf{Introduction:}
  If you are familiar with TwitchPlaysPokemon, feel free to skip this section.

  TwitchPlaysPokemon is a social experiment based on the game Pokemon, which you are probably familiar with.
  Twitch, which is a website for streaming games, has a chat service where individuals viewing the streams can chat with other viewers.
  TwitchPlaysPokemon is a stream which takes messages entered as chat as input commands into a game of Pokemon Red.

  The result? Around 70,000 individuals playing a single player game of Pokemon all together, 
  entering several hundred commands every second.
  As you can imagine, it is utterly and delightfully chaotic.

  This is a link to the \href{http://www.twitch.tv/twitchplayspokemon}{live stream}.

  \textit{Anarchy vs. Democracy:}

  As it stands, there are two modes to the game, anarchy and democracy.
  Anarchy is blissful chaos where all the commands from chat are fed into the game to be executed.
  Democracy aims at more structured approach where it gives the viewers a 20 second window to enter the commands; 
  it will then choose the command entered the most.
  The game shifts between the two modes based on votes it receives for the modes, which are also entered in chat.

  \textbf{A new mode: The Republic}

  The creator of TwitchPlaysPokemon has decided that he wants to add a third mode: Republic (or, if you like, Anarcho-Democratic Republican Theocracy). 
  In this mode, the viewer base will be split into four anarchists, two democrats, the elected president, and the High Priest of the Sacred Helix Fossil. 

  In order to maintain a balance of powers, no one faction can determine what move to make (even if, for example, all four anarchists agree on an action). 
  Requiring all eight representatives of the Republic to agree is certainly a sufficient condition to make a move, but it seems excessive. 
  With this in mind, if an entire faction, plus at least one representative from a different faction, agrees to make a certain move, then it shall be done.

  More explicitly, some examples: 
  Only four anarchists agreeing on the next move is not enough. 
  Only two democrats agreeing is not enough. 
  Only the elected President is not enough. 
  Only the High Priest is not enough. 
  All four anarchists and one democrat agreeing is enough to execute the next move.
  Both democrats and the President agreeing is enough.
  And so on.

  The creator of TwitchPlaysPokemon has hired your services to help him come up with a secret sharing scheme that accomplishes this task, summarized by the following points:

  \begin{enumerate}
  \item
    The Republic consists of four anarchists, two democrats, the elected President, and the High Priest of the Sacred Helix Fossil.

  \item
    There are secret instructions for executing the next move that needs to be known only if enough representatives of the Republic are in agreeance.
    The instructions must remain unknown to everyone (except the creator) if not enough representatives of the Republic are in agreeance.

  \item
    If only the representatives of one faction are in agreeance, the instructions remain a secret.

  \item
    If all the representatives of one faction are in agreeance plus at least one additional representatives,the instructions can be determined.

  \item
    Other combinations of representatives (e.g. two anarchists and a democrat) can either determine the instructions or be unable to do so (it is up to your discretion). 
  \end{enumerate}

\end{qunlist}

\end{document}
