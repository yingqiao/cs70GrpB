\documentclass[]{article}
% packages
\usepackage{../../cs70}
\usepackage{../../markup}
\usepackage{enumerate}
%% \usepackage{framed}
%% \usepackage{MnSymbol}
%% \usepackage{epstopdf}
\usepackage{color}
%% \usepackage[]{amsmath}
%% \usepackage{graphicx}
%% \usepackage{amssymb}
%% \usepackage{parskip}
%% \usepackage{rotating}
%% \usepackage{float}
%% \usepackage{multirow}
%% \usepackage{subcaption}
%% \usepackage{indentfirst}
%% \usepackage[left=1.5in, right=1.0in, top=1.0in, bottom=1.0in]{geometry}

\newif\ifsolutions
%\solutionstrue
\solutionsfalse %flag for solutions

\renewcommand{\answer}[1]{{\color{mydarkblue}\textbf{}#1}}
\definecolor{mydarkblue}{rgb}{0,0.25,1}

\def\title{Homework 6}

\begin{document}

\maketitle
\config{hwnum}{6}
\config{homework-due}{03/03/2014 13:00}
\config{grades-due}{03/10/2014 13:00}
\vspace{0.5em}
{\Large{\textbf{This homework is due Mar 3 2014, at 12:00 noon.}}}

\begin{qunlist}
  
\qns{RSA - 1} % Ian
\begin{enumerate}
\qpart
\item
\ifsolutions{ \answer 
{
}}\fi


\ifsolutions{ \answer {
\textbf{Motivation}
}}\fi
\end{enumerate}

  
\qns{RSA - 2} % Sara
\begin{enumerate}
\qpart
\item
\ifsolutions{ \answer 
{
}}\fi


\ifsolutions{ \answer {
\textbf{Motivation} 
}}\fi 
\end{enumerate}


\qns{Poly - 1} % Sara

\begin{enumerate}
\qpart
\item

\ifsolutions{ \answer 
{
}}\fi


\ifsolutions{ \answer {
\textbf{Motivation}
}}\fi
    
\end{enumerate}

\qns{Poly - 2} % Ying, cyc

\begin{enumerate}
\qpart
\item 

\end{enumerate}


\qns{Poly - 3} % Ying, cyc
\qpart


\qns{Poly+SS - 4} % Ying
\qpart


\qns{SS+SM - 1} % Ian


In this question, we will consider using parts of a stable marriage instance to share secrets. 

\begin{enumerate}
\qpart
\item  If the secret we want to share is coded as a number representing the final stable pairing reached when we run the traditional propose-and-reject algorithm for a instance of $n$ men and $n$ women, how big of a number can we get? (answer: there are $n!$ total pairs).

\end{enumerate}

\qns{ECC} % Ian
\qpart


\qns{Write your own problem} \\
Write your own problem related to this week's material and solve it. 
You may still work in groups to brainstorm problems, 
but each student should submit a unique problem.  
What is the problem? How to formulate it? 
How to solve it? What is the solution?
  
  
    
\end{qunlist}


\end{document}
