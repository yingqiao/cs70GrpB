\documentclass[]{article}
% packages
\usepackage{../../cs70}
\usepackage{../../markup}
\usepackage{enumerate}
\usepackage{hyperref}
%% \usepackage{framed}
%% \usepackage{MnSymbol}
%% \usepackage{epstopdf}
\usepackage{color}
%% \usepackage[]{amsmath}
%% \usepackage{graphicx}
%% \usepackage{amssymb}
%% \usepackage{parskip}
%% \usepackage{rotating}
%% \usepackage{float}
%% \usepackage{multirow}
%% \usepackage{subcaption}
%% \usepackage{indentfirst}
%% \usepackage[left=1.5in, right=1.0in, top=1.0in, bottom=1.0in]{geometry}

\newif\ifsolutions
\solutionstrue
%\solutionsfalse %flag for solutions

\renewcommand{\answer}[1]{{\color{mydarkblue}\textbf{Solution:}#1}}
\definecolor{mydarkblue}{rgb}{0,0.25,1}

\def\title{Homework 6}

\begin{document}

\maketitle
\config{hwnum}{6}
\config{homework-due}{03/03/2014 13:00}
\config{grades-due}{03/10/2014 13:00}
\vspace{0.5em}
{\Large{\textbf{This homework is due Mar 3 2014, at 12:00 noon.}}}

\begin{qunlist}
  
\qns{Tweaking RSA}\\    % Ian
In this problem we will consider several tweaks to RSA.
\begin{itemize}
\qpart
\item [a)] Warm up: RSA messages are comprised of bit strings encoding the message we want to send. These bit strings are then treated as unsigned numbers. One method of encoding a bit string as a number suggests that each $i$th bit represents whether we add the number $2^{i-1}$ or not. For example, "$1011$" can be represented as $2^3 + 2^1 + 2^0 = 11$ or as $2^0 + 2^2 + 2^3 = 13$ depending on what we interpret as the first bit. How many different messages can we encode with an $n$-bit binary string?
\qpart
\item [b)] Warm up: Let's say we choose a different method of generating numbers. What if instead we take a $n$-bit string and generate a number, $C$ as suggested above, and then further generate a number $2^C \bmod 10$. How many different messages can we encode such that each binary string gets translated into a different number?
\qpart
\item [c)] We see from the last part that we aren't able to encode $10$ different numbers. Let's assume we want to make the previous method work. We encode an $n$-bit binary string as a number, $C$, in the range of $[0, \ 2^n -1]$. We then generate a number $a*C \bmod {2^n}$. What should $a$ be such that we can encode all $n$-length binary bit strings?
\qpart
\item [d)] You want to use the RSA encryption method so that your friend can tell you a secrect. However, you realize that the two largest primes you have multiplied together are still less than ideal. Explain how you can modify the RSA encryption method to work with three primes, and thus you can send a larger secret.
\qpart
\item [e)] Your friend Marty sends you his favorite TA using the standard RSA (two large primes) encryption method; however, you realize that Marty's algorithm to calculate exponentiation in modular arithmetic has an off by one error. Instead of sending you $x^e \bmod{N}$ he has sent you $x^{e/2} \bmod{N}$. How can you recover the original message simply by changing your private key? Prove that this works.
\end{itemize}


\ifsolutions{ \answer { 
\begin{enumerate}
\item[a)] $2^n$. In an ideal mapping, each binary string will encode a different number (or message). So, counting the number of messages amounts to counting the number of binary strings of length $n$. We realize that each bit has $2$ choices, and there are $n$ bits, thus there are $2^n$ different ways to choose a binary string.
\item[b)] $5$. We know that $2$ is a factor of $10$ so we will not be able to reach every number simply by taking successive powers of $2$. You can check that the powers repeat themselves every time we multiply by $2^4$. \\ $\forall{i} \in \mathbb{N}:   (2^{1+4i} = 2 \bmod{10}) \land (2^{2+4i} = 4\bmod{10}) \land (2^{3+4i} = 8\bmod{10}) \land (2^{4+4i} = 6\bmod{10})$. So, we can only reach these four numbers by repeatedly multiply by 2. However, if $C=0$, then $2^C = 1$ so we can actually encode 5 numbers.

\item[c)] There are many possible values for $a$. In particular, we want $a$ to be co-prime with $2^n$. This means if $a$ is odd, it will work. We want $a$ to be co-prime with $2^n$ so that an inverse exists. We argue that $a\cdot C$ is a bijection from the numbers $\bmod{2^n}$ to the numbers $\bmod{2^n}$. We use the lemma in lecture note 6, $f: A \rightarrow A$ is a bijection if there is an inverse function $g: A \rightarrow A$ such that $\forall{x} \in A:  g(f(x)) = x$. We have $f(C) = a\cdot C \bmod{2^n}$. Since $a$ is coprime with $2^n$ it has an inverse, $a^{-1}$. let $g(x) = a^{-1} \cdot x \bmod{2^n}$. Then we have $g(f(C)) = a^{-1} \cdot a \cdot C = C$. Therefore, $f$ is a bijection and we can encode all $n$-length binary bit strings.

\item[d)] $N = pqr$ where $p, q, r$ are all prime. Then, let $e$ be co-prime with $(p-1)(q-1)(r-1)$. Give the public key: $(N, e)$ and calculate $d = e^{-1} \bmod{(p-1)(q-1)(r-1)}$. People who wish to send me a secret, $x$, send $y = x^e \bmod{N}$. I decrypt an incoming message, $y$, by calculating $y^d \bmod{N}$. 

Does this work? We prove that $x^{ed} - x \equiv 0 \bmod{N}$ and thus $x^{ed} = x \bmod{N}$. To prove that  
$x^{ed} - x \equiv 0 \bmod{N}$, we factor out the $x$ to get \\ $x\cdot (x^{ed-1} -1) = x\cdot (x^{k(p-1)(q-1)(r-1)}-1)$ because $ed \equiv 1 \bmod{(p-1)(q-1)(r-1)}$.
%Proceeding from before we know that $x^{ed} = x*(x^{k*(p-1)*(q-1)*(r-1) } -1)$ 
We now argue that this number must be divisible by $p$, $q$, and $r$. Thus it is divisible by $N$ and $x^{ed} -x \equiv 0 \bmod{N}$. \\
To prove that it is divisible by $p$:
\begin{itemize}
\item if $x$ is divisible by $p$, then the entire thing is divisible by $p$. 
\item if $x$ is not divisible by $p$, then that means we can use FLT on the inside to show that $(x^{p-1})^{k(q-1)(r-1) } -1 \equiv 1 -1 \equiv 0 \bmod{p} $. Thus it is divisible by $p$. 
\end{itemize} 
To prove that it is divisible by $q$:
\begin{itemize}
\item if $x$ is divisible by $q$, then the entire thing is divisible by $q$. 
\item if $x$ is not divisible by $q$, then that means we can use FLT on the inside to show that $(x^{q-1})^{k(p-1)(r-1) } -1 \equiv 1 -1 \equiv 0 \bmod{q} $. Thus it is divisible by $q$. 
\end{itemize} 
To prove that it is divisible by $r$:
\begin{itemize}
\item if $x$ is divisible by $r$, then the entire thing is divisible by $r$. 
\item if $x$ is not divisible by $r$, then that means we can use FLT on the inside to show that $(x^{r-1})^{k(p-1)(q-1) } -1 \equiv 1 -1 \equiv 0 \bmod{r} $. Thus it is divisible by $r$. 
\end{itemize} 

\item[e)] Calculate $d$ such that $de \equiv 2 \bmod{N}$. In other words, $d \equiv (2^{-1}e)^{-1} \equiv 2e^{-1} \bmod{N}$. We wish to prove that $x^{ed} = x^2 \bmod{N}$. We now wish to show that $x^{ed} - x^2 = 0 \bmod{N}$. We now factor out $x^2$ to get $x^2(x^{ed-2} - 1)$. Because $ed \equiv 2 \bmod{(p-1)(q-1)}$, we know that $ed = 2 + k(p-1)(q-1)$. Therefore, \\ $x^{ed} - x^2 = x^2(x^{2+ k(p-1)(q-1) -2} - 1) = x^2(x^{k(p-1)(q-1)} - 1)$. We argue that this number is divisible by both $p$ and $q$, and is thus divisible by $N$ and then $\equiv 0 \bmod{N}$. If $x$ is divisible by $p$, we know that $x^2$ is divisible by $p$, and thus the whole thing is divisible by $p$. If $x$ is not divisible by $p$, we can use FLT in the inside to see it is equal to $0 \bmod{p}$. In essence the proof continues as in the lecture notes for normal RSA. 
\end{enumerate}
}}\fi


%\textbf{Motivation} Parts d is to make sure students really understand the proof of RSA. Part e is for students to gain some intuition about how to actually go about calculating $e$ and $d$ and why the work. Part a,b serve as warm ups for the problem set and introduce them to counting which they will see a lot more in the second half of the course. Part c is to make sure students are actually learning from the midterms/homeworks, rather than just forgetting everything.



  



\qns{Solving systems of equations}\\   % Sara
% POLY problem,  reuse from sp13, numbers have been adjusted, please check solution!
Consider the following system of equations $\bmod{21}$:
\begin{align*}
16x &+ 5y &= 14 \\
9x &+ 11y &= 4
\end{align*}

\begin{itemize}  
\qpart
\item[a)] Decompose this into two systems of equations, one $\bmod 3$ and one $\bmod 7$.
\qpart
\item[b)] Solve for $x$ and $y$ in both systems.
\qpart
\item[c)] Combine these solutions to get a solution to the original problem.
\end{itemize}


\ifsolutions{ \answer {
Given a system of 2 linear equations in two unknowns, we are strongly tempted to just attempt to solve it by traditional means. However, we have a problem. Some of these coefficients are not relatively prime to 21. So, we can't necessarily solve for one variable in terms of the other and then use substitution. Looking at $21=3 \cdot 7$ lets us decompose into two different sets of equations, each pair of which is in a finite field where division by nonzero numbers is allowed and safe.
\begin{enumerate}
\item[a)] 
$\bmod 3$:
  \begin{align*}
  1x &+ 2y &= 2 \\
  0x &+ 2y &= 1 
  \end{align*}
$\bmod 7$:
  \begin{align*}
  2x &+ 5y &= 0 \\
  2x &+ 4y &= 4 \\
  \end{align*}

\item[b)]
$\bmod 3$:\\
We find that $y=2 \bmod 3$ from the second equation, since $2^{-1} \bmod 3 = 2$. We can substitute this into the first equation to find $x$:
\[
x + 2\cdot2 = 2 \quad \Rightarrow \quad x = -2 \bmod 3 \quad \Rightarrow \quad x = 1 \bmod 3.
\]
$\bmod 7$:\\
We can subtract the two equations to find $y$:
\begin{align*}
2x &+ 5y &= 0 \\
(-) \quad 2x &+ 4y &= 4 \\
\hline
\Rightarrow \quad  & \quad \ \ y &= -4
\end{align*}
Since $-4 = 3\bmod 7$, this gives us $y = 3\bmod 7$. Substituting and reducing, we find that:
\[ 2x + 5\cdot 3 = 0 \quad \Rightarrow \quad 2x = -15 \bmod 7 = 6 \bmod 7 \quad \Rightarrow \quad x = 3\bmod 7. \]

\item[c)] Now, we want to find values of $x$ and $y$ such that they satisfy the above congruences $\bmod 3$ and $\bmod 7$. So let us construct a solution accordingly. Let us first define the following notation $p^{-1}_q$ to be the inverse of $p, \bmod q$.
Now, look at $x$. Consider the following form of $x$, inspired by Lagrange interpolation:
\[
x = 1 \cdot (7 \cdot 7^{-1}_3) + 3 \cdot (3 \cdot 3^{-1}_7)
\]
Does this satisfy our congruences? We see that when we take the equation $\bmod 3$, the second term disappears since it is congruent to $0 \bmod 3$. The remaining term has both $7$ and its inverse modulo $3$. They give $1$ together and we are left with the original value of $x\bmod 3$, which is 1. We see the same thing happening in the case of $\bmod 7$. $3$ and its inverse give $1$ and we are left with just $3\bmod 7$. Expanding $x$ and taking it $\bmod 21$, we get:
\begin{align*}
x = 1 \cdot (7 \cdot 7^{-1}_3) + 3 \cdot (3 \cdot 3^{-1}_7) &= 1 \cdot 7 \cdot 1 + 3 \cdot 5 \cdot 3 \\
&= \boxed{10 \pmod {21}}
\end{align*}
%      Is this unique$\pmod{15}$? We can show that it is as follows:\\
%      Consider some other $x'$ which satisfies the same congruences as $x$. Therefore, for some integers $k_1$ and $k_2$, we have
%      \begin{align*}
%	x = x'\pmod 3 \Rightarrow x - x' = 3k_1\\
%	x = x'\pmod 5 \Rightarrow x - x' = 5k_2
%      \end{align*}
%      But, since 3 and 5 are prime, we have $x - x' = 15k_3$ for some other integer $k_3$.
%      \\This shows that $x \equiv x'\pmod{15}$ which is what we wanted to show.

Similarly, for $y$ we get
\begin{align*}
y = 2 \cdot (7 \cdot 7^{-1}_3) + 3 \cdot( 3 \cdot 3^{-1}_7) &= 2 \cdot 7 \cdot 1 + 3 \cdot 5 \cdot 3 \\
&= \boxed{17 \pmod{21}} \\
\end{align*}

%      \textit{Alternate method outline:}\\
%      Consider the congruences of $x\pmod 3$ and$\mod 5$. We can write $x$ as follows:
%      $$x = 3t_1 = 5t_2 + 2$$
%      Since $t_1$ and $t_2$ have to be integers, we can solve $t_1 = \frac{5t_2 + 2}{3}$ for possible values of $t_1$ and $t_2$.\\
%      We get that the above equation has integer solutions for $t_1$ and $t_2$ only if $t_2$ is of the form $3k - 1$ for some integer $k$. Then, this forces $t_1$ to be of the form $5k-1$.\\
%      So, we find that $x$ is of the form $15k-3$ for some $k$. Thus, $x \equiv -3 \equiv 12\pmod{15}$. Further, we see that this is unique$\pmod{15}$ because of the $15k$ term.\\\\
%      We can do the same for $y$ to get $y = 8\pmod{15}$.\\\\
  \end{enumerate}
}}\fi




\qns{Polynomial Interpolations} % Ying
% POLY problem,  reuse from fa11 and fa10, combined
\begin{itemize}
\qpart
\item[a)] Consider the set of four points $\{(0,1), (1,-2), (3,4), (4,0)\}$, construct the unique degree-$3$ polynomial (over the reals) that passes through these four points by writing down and solving a system of linear equations.


\ifsolutions{ \answer {
Suppose the unique degree $3$ polynomial passing through the four given points is 
\[
p(x) = a_0 + a_1 x + a_2 x^2 + a_3 x^3
\]  
The coefficients of $p(x)$ satisfy the following linear equations:
\begin{align}
  p(0) = 1 &\Rightarrow a_0 = 1 \label{p(0)} \\
  p(1) = -2 &\Rightarrow a_0 + a_1 + a_2 + a_3 = -2 \label{p(1)} \\
  p(3) = 4 &\Rightarrow a_0 + 3a_1 + 9a_2 + 27a_3 = 4 \label{p(3)} \\
  p(4) = 0 &\Rightarrow a_0 + 4a_1 + 16a_2 + 64a_3 = \label{p(4)} 0
\end{align}
Substituting $a_0 = 1$ and subtracting (i) $3$ times equation \eqref{p(1)} from equation \eqref{p(3)} and (ii) 4 times equation \eqref{p(1)} from equation \eqref{p(4)}, we obtain the following simultaneous equations:
\begin{align*}
  6a_2 + 24a_3 &= 12 \\
  12a_2 + 60a_3 &= 11
\end{align*}
Solving for $a_2$ and $a_3$, we obtain $a_2 = 76/12$ and $a_3 = -13/12$. Substituting in equation \eqref{p(1)} we obtain $a_1 = -99/12$. Hence 
\[
p(x) = (12 - 99x + 76x^2 - 13x^3)/12
\] 
is the unique degree $3$ polynomial passing through the given points.
}}\fi



\qpart
\item[b)] Use Lagrange interpolation to find a polynomial $p(x)$ of degree at most $2$ that passes through the points $(1,2)$, $(2,3)$, and $(3,5)$, working in $GF(7)$. In other words, we want $p(x)$ to satisfy
$p(1)\equiv 2 \pmod 7$, 
$p(2)\equiv 3 \pmod 7$, and
$p(3)\equiv 5 \pmod 7$.
Show your work clearly and use the same notations as in Lecture Note 7.


\ifsolutions{ \answer {
First we would find the three $\Delta_i(x)$ polynomials.
\begin{align*}
\Delta_1(x) &\equiv \frac{(x-2)(x-3)}{(1-2)(1-3)} \equiv \frac{(x-2)(x-3)}{2} \equiv 4(x-2)(x-3) \equiv 4x^2 + x + 3  \pmod 7 \\
\Delta_2(x) &\equiv \frac{(x-1)(x-3)}{(2-1)(2-3)} \equiv \frac{(x-1)(x-3)}{-1} \equiv -(x-1)(x-3) \equiv 6x^2 + 4x + 4 \pmod 7\\
\Delta_3(x) &\equiv \frac{(x-1)(x-2)}{(3-1)(3-2)} \equiv \frac{(x-1)(x-2)}{2} \equiv 4(x-1)(x-2) \equiv 4x^2 + 2x + 1 \pmod 7
\end{align*}

Next we compute $p(x)$:
\begin{align*}
p(x) &\equiv 2 \Delta_1(x) + 3 \Delta_2(x) + 5 \Delta_3(x)  \pmod 7\\
&\equiv 2(4x^2+x+3)+3(6x^2+4x+4)+5(4x^2+2x+1)\pmod 7\\
&\equiv x^2+2x+6+4x^2+5x+5+6x^2+3x+5 \pmod 7\\
&\equiv 4x^2+3x+2 \pmod 7
\end{align*}

%fa09 old solution format
%\begin{eqnarray*}
%\Delta_2(x)
%&\equiv&\frac{(x - 0)(x - 1)}{(2 - 0)(2 - 1)} \pmod{7}\\
%&\equiv&2^{-1} \cdot (x^2 - x) \pmod{7}\\
%&\equiv&4 \cdot (x^2 - x) \pmod{7}\\
%&\equiv&4x^2 - 4x \pmod{7}\\
%&\equiv&4x^2 + 3x \pmod{7}
%\end{eqnarray*}

}}\fi

\end{itemize}





\qns{GCD of Polynomials} \\   % Ying
% POLY problem,  reuse problem from fa13, new solution
Let $A(x)$ and $B(x)$ be polynomials (with coefficients in $\R$ or $GF(m)$). We say that  $\gcd{A(x), B(x)} = D(x)$ if $D(x)$ divides $A(x)$ and $B(x)$, and if every polynomial $C(x)$ that divides both $A(x)$ and $B(x)$ also divides $D(x)$. For example, $\gcd{(x-1)(x+1), (x-1)(x+2)} = x-1$. Incidentally, $\gcd{A(x), B(x)}$ is the highest degree polynomial that divides both $A(x)$ and $B(x)$.
%(and it is only defined up to multiplication by a constant, i.e. we could also say $\gcd{(x-1)(x+1), (x-1)(x+2)} = 2x-2$).

\begin{itemize}
\qpart
\item[a)] Write a recursive program to compute $gcd(A(x),B(x))$. 
You may assume you already have a subroutine for dividing two polynomials. 


\ifsolutions{\answer{
Consider the following recursive definition $F$:
\begin{itemize}
\item $F(A(x), 0) = A(x)$.
\item If $A(x) = B(x) Q(x) + R(x)$ with $\deg{R(x)} < \deg{B(x)}$, then
\[
F(A(x), B(x)) = F(B(x), R(x)).
\] 
($\deg{P(x)}$ denotes the degree of $P(x)$.) 
\end{itemize}
}}\fi

\end{itemize}


Let $P(x) = x^4 - 1$ and $Q(x) = x^3 + x^2$ in standard form.
\begin{itemize}
\qpart
\item[b)] Prove there are no polynomials $A(x)$ and $B(x)$ such that $A(x) P(x) + B(x) Q(x) = 1$ for all $x$.

\ifsolutions{ \answer {
$gcd(P(x), Q(x)) = x+1$
}}\fi


\qpart
\item[c)] Find polynomials $A(x)$ and $B(x)$ such that $A(x) P(x) + B(x) Q(x) = x + 1$ for all $x$. 

\ifsolutions{ \answer {
\[
A(x)=-(x+1) \quad B(x)=x^2
\]
}}\fi

\end{itemize}






\qns{Proofs about polynomials} \\   % Ian
%reuse from su13 and sp10 similar to hw5 q3 (no .tex in repo)
In this problem, you will give two different proofs of the following theorem: For every prime $p$, every polynomial over $GF(p)$, even polynomials with degree $\geq p$, is equivalent to a polynomial of degree at most $p-1$. (Two polynomials $f,g$ over $GF(p)$ are said to be equivalent iff $f(x)=g(x)$ for all $x\in GF(p)$.)

\begin{enumerate}
\qpart
\item[a)] Show how the theorem follows from Fermat's Little Theorem. 
%(Hint: Be careful! It is not true that $x^{p-1}\equiv 1\pmod{p}$ for all $x\in\{0,1,\ldots,p-1\}$. Why not?)
\qpart
\item[b)] Now prove the theorem using what you know about Lagrange interpolation.
\end{enumerate}


\ifsolutions{ \answer {
\begin{itemize}
\item[a)]
From Fermat's Little Theorem, we know $\forall x \not\equiv 0, \text{ } x^{p-1} \equiv 1 \pmod{p}$. \\
Multiplying both sides by $x$, and noting that $0^p \equiv 0 \pmod{p}$, we can see that 
\[
\forall x, \text{ } x^p \equiv x \pmod{p}
\]
Therefore, any $x^k$, where $k \geq p$ will be equivalent to $x^n$, where $n \in \{0, 1, \dots, p-1\}$, and will have a degree at most $p - 1$. 

\item[b)]
Since a polynomial $f$ of degree $d$ is described completely by $d + 1$ points, we cannot specify a polynomial of degree $\geq p$ because $f(p) = f(0)$ (and so forth for other values larger than $p$) over $GF(p)$. Thus, we can only specify at most $p$ unique points. Therefore, every polynomial is equivalent to a polynomial of degree at most $p - 1$.
\end{itemize}
}} \fi





\qns{Poker mathematics}\\   %Ying
% SS problem, reuse from fa10
A \emph{pseudo-random number generator} is a way of generating a large quantity of random-looking numbers, if all we have is a little bit of randomness (known as the \emph{seed}). One simple scheme is the \emph{linear congruential generator}, where we pick some modulus $m$, some constants $a,b$, and a seed $x_0$, and then generate the sequence of outputs $x_0,x_1,x_2,x_3,\dots$ according to the following equation:
\[ 
x_{t+1} = \bmod(ax_t + b, m) 
\]
(Notice that $0 \le x_t < m$ holds for every $t$.)

You've discovered that a popular web site uses a linear congruential generator to generate poker hands for its players.  For instance, it uses $x_0$ to pseudo-randomly pick the first card to go into your hand, $x_1$ to pseudo-randomly pick the second card to go into your hand, and so on. For extra security, the poker site has kept the parameters $a$ and $b$ secret, but you do know that the modulus is $m=2^{31}-1$ (which is prime).

Suppose that you can observe the values $x_0$, $x_1$, $x_2$, $x_3$, and $x_4$ from the information available to you, and that the values $x_5,\dots,x_9$ will be used to pseudo-randomly pick the cards for the next person's hand. Describe how to efficiently predict the values $x_5,\dots,x_9$, given the values known to you.


\ifsolutions{ \answer { 
\\ \textbf{Answer 1:}
We know 
\begin{eqnarray*}
 x_1 &\equiv& a x_0 + b \pmod m \\
 x_2 &\equiv& a x_1 + b \pmod m 
\end{eqnarray*}
Because we know $x_0$, $x_1$, and $x_2$, this is a system of two equations with two unknowns (namely, $a$ and $b$). So we can solve for $a$ and $b.$  More explicitly, by subtracting the first equation from the second, we get 
\begin{equation*}
x_2 - x_1 \equiv a (x_1 - x_0) \pmod{m}
\end{equation*}
If $x_0 = x_1$ then by induction on $n$ we see $x_n = x_0$ for all $n$ which allows us to immediately  calculate $x_5, x_6, x_7, x_8$ and $x_9.$ So suppose  $x_0 \neq x_1$.  Then $x_1 - x_0$ is invertible modulo $m$ (because $m$ is prime, therefore  $\gcd(x_0-x_1, m)=1$), and we see
\begin{equation*}
a \equiv (x_2-x_1)(x_1-x_0)^{-1} \pmod{m}
\end{equation*}
Once we know $a$, we can plug in the known value of $a$ into the first equation and solve for $b$:
\begin{equation*}
b \equiv x_1 - ax_0 \equiv x_1 - x_0 (x_2-x_1)(x_1-x_0)^{-1} \pmod m
\end{equation*}
Since we know $a$ modulo $m$ and $b$ modulo $m$, we can compute $x_5, x_6, x_7, x_8$ and $x_9.$

\bigskip\textbf{Answer 2:}
Alternatively, we could start by solving for $b$.  Multiplying the first equation by $x_1$, multiplying the second equation by $x_0$, and subtracting gives $x_1^2 - x_2 x_0 \equiv b (x_1-x_0) \pmod{m}$, and then
\begin{equation*}
b \equiv (x_1^2 - x_0 x_2)(x_1 - x_0)^{-1} \pmod{m}
\end{equation*}
Now we can plug in the known value of $b$ into the first equation and solve for $a$,  and continue as before.

\bigskip{\bf Note:} For both answers, It is important to consider the case when $x_1 \equiv x_0 \pmod m$. Solutions that didn't address that were incomplete, as $x_1 - x_0$ would not necessarily have an inverse modulo $m$.

\bigskip{\bf Comment:}
This homework exercise was loosely modelled after a real-life story, where a group of computer scientists discovered serious flaws in the pseudorandom number generator used by several real poker sites. They could have used this to make thousands of dollars off everyone else who played at that site, but rather than cheat, they instead revealed the flaw to the poker site and the public. For more, see their paper
``How We Learned to Cheat at Online Poker: A Study in Software Security''
(\url{http://www.cigital.com/papers/download/developer_gambling.php}).

}}\fi



\qns{Twitch Plays Pokemon ... a little differently}  % Sara + Sibi 
% RSA+SS problem, adapt into Twitch plays Pokemon scenario, break into different parts

% Notes from original source (reuse from sp 13)
%I want it to show how to compose secret-sharing schemes hierarchically. So break the secret into two shares --- a unanimity share and a multirace share. Both are required to get the true secret. Then, have the unanimity share be again divided into pieces depending on the race. Only when all the members of a race are together can they recover the u key. Have the multirace key be shared as a degree 1 polynomial evaluated once for every race. Each hobbit has the same copy of this piece, etc... This two layer process works. If no race is unanimous, they will never get the u key. If only one race is involved, they won't get the m key. Without both u and m keys, they have nothing.

\textbf{Introduction:} If you are familiar with TwitchPlaysPokemon, feel free to skip this section.

TwitchPlaysPokemon is a social experiment based on the game Pokemon, which you are probably familiar with.Twitch, which is a website for streaming games, has a chat service where individuals viewing the streams can chat with other viewers. TwitchPlaysPokemon is a stream which takes messages entered in chat as input commands into a game of Pokemon Red.

The result? Around 70,000 individuals playing a single player game of Pokemon all together, entering several hundred commands every second. As you can imagine, it is utterly and delightfully chaotic. This is a link to the \href{http://www.twitch.tv/twitchplayspokemon}{live stream}.

\textit{Anarchy vs. Democracy:}
As it stands, there are two modes to the game, anarchy and democracy. Anarchy is blissful chaos where all the commands from chat are fed into the game to be executed. Democracy aims at more structured approach where it gives the viewers a 20 second window to enter the commands; it will then choose the command entered the most. The game shifts between the two modes based on votes it receives for the modes, which are also entered in chat.

%\newpage
\textbf{A new mode: The Republic}

The creator of TwitchPlaysPokemon has decided that he wants to add a third mode: Republic (or, if you like, Anarcho-Democratic Republican Theocracy). 
In this mode, the viewer base is reduced in number slightly from around 70,000 to $8$ representatives. 
These eight representatives are split into four anarchists, two democrats, the elected president, and the High Priest of the Sacred Helix Fossil. 

In order to maintain a balance of powers, no one faction can determine what move to make (even if, for example, all four anarchists agree on an action). Requiring all eight representatives of the Republic to agree is certainly a sufficient condition to make a move, but it seems excessive. With this in mind, if an entire faction, plus at least one representative from a different faction, agrees to make a certain move, then it shall be done.
The creator of TwitchPlaysPokemon has hired your services to help him come up with a secret sharing scheme that accomplishes this task.

More explicitly, some examples: 
Only four anarchists agreeing on the next move is not enough. 
Only two democrats agreeing is not enough. 
Only the elected President is not enough. 
Only the High Priest is not enough. 
All four anarchists and one democrat agreeing is enough to execute the next move.
Both democrats and the President agreeing is enough.
And so on.

It is of course possible that multiple factions could recover the secret instructions simultaneously.
Consider for example: All the democrats and one anarchist come together and decide a move. 
At the same time, the President and one other anarchist decide a move together. 
And finally, the High Priest and the remaining anarchists decide another move together. 
Here, these individiual groups can all recover the secret instructions from the specified conditions.
If such a rivalry arises, then, true to the original spirit of the game, there is fallback to anarchy: one of the multiple moves is chosen at random.
You do not need to concern yourself with this outcome.

The secret sharing scheme is summarized by the following points:

\begin{enumerate}
\item  The Republic consists of four anarchists, two democrats, the elected President, and the High Priest of the Sacred Helix Fossil.
\item  There are secret instructions for executing the next move that needs to be known only if enough representatives of the Republic are in agreeance. The instructions must remain unknown to everyone (except the creator) if not enough representatives of the Republic are in agreeance.
\item  If only the representatives of one faction are in agreeance, the instructions remain a secret.
\item  If all the representatives of one faction are in agreeance plus at least one additional representatives, the instructions can be determined.
\item  Other combinations of representatives (e.g. two anarchists and a democrat) cannot determine the instructions.
\end{enumerate}


Now, please devise the required secrect sharing scheme using the guided parts given below.

\begin{enumerate}
\qpart
\item[(a)] Devise a scheme that will ensure the representatives of a faction agree within themselves.  Give an example of how your method will work when the secret $s=4$.
\qpart
\item[(b)] Then devise a scheme that will ensure that at least two factions agree with each other. Extend your example from part (a) to this step.
\qpart
\item[(c)] Finally, show how to combine (a) and (b) so that they can recover the secret. In your example, show how they will recover $s=4$.
\end{enumerate}




\ifsolutions{ \answer 
{

There will be two parts to this secret: a unanimity secret $U$ and a multi-faction secret $M$. $U$ ensures that at least all members of one faction are in agreement while $M$ ensures that members of at least two races are in agreement. Once both $U$ and $M$ are recovered, they can then be combined to reveal the original secret: each will be a point of the degree-1 polynomial $R(x)$ whose $y$-intercept contains the secret instructions. \\
As an example, suppose the secret is $s=4$, $M=3$, and $U=2$. From now on, we can work in GF(5) since $s<5$ and $n<5$ ($n$ is the number of people who have pieces of the secret). First we need to create a degree-1 polynomial $R(x)$ such that $R(0)=s=4$, $R(1)=M=3$, and $R(2)=U=2$. By inspection, 
\[ R(x) = 4x + 4 \]
has these properties, since $R(0) = 4 \bmod 5$, $R(1) = 4\cdot1 + 4 = 8 \equiv 3 \bmod 5$, and $R(2) = 4 \cdot 2 + 4 = 12 \equiv 2 \bmod 5$.

\begin{enumerate}

\item[(a)]

The \textit{unanimity secret} involves creating a separate secret for each faction. We will require all members of that faction to join forces in order to reveal the secret. For example, the anarchists will each have distinct points of a degree-3 polynomial and the democrats will each have distinct points of a degree-1 polynomial. When all members of a faction come together, they will reveal $U$ (encoded, for example, as the $y$-intercept of each of these polynomials). Note that the President and the High Priest will each be told $U$ directly since they are the only members of their faction.
%When creating the unanimity secret $U$, we first note that each of the dwarf and the elf will be told $U$ directly since they are the only members of their respective races.
On the other hand, the democrats will each have a point on the degree-1 polynomial $P_d(x)$. Suppose $P_d(x) = 2x+2$. Then the first democrat receives $P_d(1) = 4$ and the second receives $P_d(2) = 4+2 = 6 \equiv 1 \bmod 5$. When they interpolate using these values, they will discover the original polynomial and therefore $P_d(0) = U = 2$. The anarchists will have a similar secret but with a degree-3 polynomial (e.g. $P_a(x) = 4x^3 + x^2 + 2$).

\item[(b)]

The \textit{multi-faction secret} involves creating a degree-1 polynomial $P_m(x)$ and giving one point to all members of each faction. For example, the anarchists can each get $P_m(1)$, the democrats each get $P_m(2)$, the President gets $P_m(3)$, and the High Priest gets $P_m(4)$. In this way if members of any two factions are in agreement, they can reveal $M$ (encoded, for example, as the $y$-intercept of $P_m(x)$). To continue our example, we choose degree-1 polynomial $P_m(x) = x+3$ and tell each anarchist $P_m(1)=4$, each democrat $P_m(2)=5\equiv0 \bmod 5$, the President $P_m(3)=6 \equiv 1 \bmod 5$, and the High Priest $P_m(4) = 7 \equiv 2 \bmod 5$. Now any two members of distinct factions can determine $P_m(x)$ and thus $P_m(0) = M = 3$ by interpolating their two values.

\item[(c)]

Once $U$ and $M$ are each known, they can be \textit{combined} to determine the final secret. $U$ and $M$ allow us to uniquely determine $R(x)$ and thus $R(0)$, the secret instructions. For example, suppose that the two democrats and the High Priest come together. The two democrats work together to determine $U=2$ as described above. Together the three of them also know $P_m(2)=0$ and $P_m(4)=2$, from which they can find $P_m(x)$ and thus $P_m(0) = M=3$. Now that they have $U$ and $M$, they can interpolate to find 
\[ R(x) = 3 \frac{(x-2)}{(1-2)} + 2 \frac{(x-1)}{(2-1)} = -3x+6 + 2x-2 = -x + 4 \equiv 4x + 4 \pmod 5 \]
and thus $R(0) = s = 4$. This scheme is an example of hierarchical secret sharing. 

\end{enumerate}




%\answer (\#2) \textbf{NO LONGER WORKS WITH THE NEW ``POINT-5'' IN SUMMARY}
%
%The trick is to construct a degree-4 polynomial (which requires 5 points to determine) $P(x)$ and reveal a varying number of points to each member of the party, depending on their race:
%
%\begin{itemize}
%	\item Hobbits: 1 point each
%	\item Humans: 2 points each
%	\item Dwarf: 4 points
%	\item Elf: 4 points
%\end{itemize}
%
%Since, we need 5 points to determine the polynomial (and thus the secret, which without loss of generality we will assume is encoded as $P(0)$), we see the following behavior (for example):
%\begin{itemize}
%	\item Four hobbits only = 4 points (not enough)
%	\item Two men only = 4 points (not enough)
%	\item Elf only = 4 points (not enough)
%	\item Dwarf only = 4 points (not enough)
%	\item Four hobbits + one man = 4 + 2 = 6 points (enough)
%	\item Two men + dwarf = 4 + 4 = 8 points (enough)
%	\item Elf + dwarf = 4 + 4 = 8 points (enough)
%	\item Three hobbits + one man = 3 + 2 = 5 points (enough) (note this means that unanimity is not guaranteed by this solution)
%	\item Two men + one hobbit = 4 + 1 = 5 points (enough)
%\end{itemize}
%
%Once enough points (at least 5) are known, the polynomial (and thus the secret) can be reconstructed via interpolation.

\textbf{Note:} If you have a different (and correct) solution, please post it on Piazza for potential extra credit.

}}\fi



\qns{How many errors?} \\ % Ian
Suppose that the message we want to send consists of 10 numbers. We find a polynomial of degree 9 which goes through all these points and evaluate it on 25 points. How many erasure errors can we recover from?

\ifsolutions{ \answer {
For $ k $ erasure errors, we need to send $ k $ additional packets for a total of $ n+k $. Here, $ n=10 $ and $ n+k=25 $, so we can recover from 15 erasure errors.

For $ k $ general errors, we need $ 2k $ additional packets. We have 15 additional packets, so we may recover from $ \lfloor 15/2 \rfloor = 7 $ general errors.
}}\fi



\qns{Write your own problem} \\
Write your own problem related to this week's material and solve it. You may still work in groups to brainstorm problems, but each student should submit a unique problem. What is the problem? How to formulate it? How to solve it? What is the solution?
  
  
    
\end{qunlist}


\end{document}
