\documentclass[11pt,fleqn]{article}
\usepackage{../cs70,latexsym,epsf,enumerate}
\usepackage{lastpage}
\usepackage{color}
\def\title{Midterm 2}

\newcommand{\fillin}[1]{\underline{\hskip #1}}
\newcommand{\doublehrule}{\hrule \vskip 0.02in \hrule}
%\newcommand{\startnewpage}{\newpage \noindent {\sc Print} your name and student ID: \fillin{5in}\\[0.3in]}
\newcommand{\startnewpage}{\newpage \noindent \vspace{0mm} {\sc Print} your name and student ID: \fillin{5in}\\[-0.2in]}

\definecolor{mydarkblue}{rgb}{0,0.25,1}
\renewcommand{\answer}[1]{{\color{mydarkblue}\textbf{Solution: }#1}}

\newcommand{\charbox}{\fbox{
	\begin{minipage}{.5cm}
		\hfill\vspace{.5cm}
	\end{minipage}
	}}


\addtolength{\parskip}{0.3\baselineskip}
\begin{document}
\maketitle

\begin{center}
\framebox{
\framebox{
\Large{
%% Note: each of the following needs to be compiled, saved as a separate file, and sent for copying.

Exam location: 1 Pimentel, back half: SIDs with second-to-last digit 1 or 3
%Exam location: 1 Pimentel, front half: SIDs with second-to-last digit 7 or 9
%Exam location: 2050 VLSB, back half: SIDs with second-to-last digit 0 or 4
%Exam location: 2050 VLSB, front half: SIDs with second-to-last digit 6 or 8
%Exam location: 2040 VLSB: SIDs with second-to-last digit 5
%Exam location: 2060 VLSB: SIDs with second-to-last digit 2
%Exam location: 540 Cory
}
}
}
\end{center}

\noindent
% \vspace{1mm} \\
{\sc Print} your student ID:   \charbox \charbox \charbox \charbox \charbox \charbox \charbox \vspace{.1mm} \fbox{\charbox} \charbox     \\   % \fillin{3in} 
\vspace{-2mm} \\
{\sc Print and Sign} your name:
\begin{tabular}{ccc}
\\
\fillin{1.3in},
&\fillin{1.3in}
&\fillin{1.6in}\\
{\small (last)}
&{\small (first)}
&{\small (signature)}\\
\end{tabular}\\[.1in]
\vspace{-1mm} \\
{\sc Print} your Unix account login: cs70-\fillin{1in}\\
\vspace{1mm} \\
%{\sc Print} where you are taking this exam: \fillin{3in}\\
%\vspace{3mm} \\
{\sc Print} your discussion section and GSI (the one you attend): \fillin{3in}
\vspace{-7mm} \\

%Names of the people sitting next to you: \fillin{4in}\\
Name of the person to your left: \fillin{4in}\\
\vspace{1mm} \\
Name of the person to your right: \fillin{4in}\\
\vspace{1mm} \\
Name of someone in front of you: \fillin{4in}\\
\vspace{1mm} \\
Name of someone behind you: \fillin{4in}




\section*{Section 0: Pre-exam questions (3points)}

\begin{qunlist}

\qns{What is your favorite song? (1 pt)}

\vspace{0.5in}

\qns{Describe a sense of accomplishment that you have felt and what
  prompted it. (2pts)}

\vspace{1.5in}

\begin{center}
\framebox{
\framebox{
Do not turn this page until the proctor tells you to do so.
%You can work on Section 0 above before time starts.
}
}
\end{center}



%Number of pages in this exam (some are blank): \pageref{LastPage}.



%\startnewpage
%
%
%{\sc Some approximations and other useful tricks that may or may not
%  come in handy:}
%
%$$n! \approx \sqrt{2\pi n} (\frac{n}{e})^n \hspace{30mm} {n \choose k} \leq (\frac{n e}{k})^k \hspace{30mm} \lim_{n\rightarrow \infty} (1+\frac{x}{n})^n = e^x$$
%
%%$${n \choose k} \leq (\frac{n e}{k})^k$$
%
%When $x$ is small, $\ln(1+x) \approx x$
%
%When $x$ is small, $(1+x)^n \approx 1+nx$
%
%%$$\lim_{n\rightarrow \infty} (1+\frac{x}{n})^n = e^x$$
%
%%You probably have others on your note sheet and in your minds. Good
%%for you.
%
%
%
%
%
%%\startnewpage
%
%    \begin{center}
%      \includegraphics[width=.9\textwidth]{normal_table.png}
%    \end{center}
%    
%    Source: \url{http://www.math.unb.ca/~knight/utility/NormTble.htm}




%%%%%%%%%%%%%%%%%%%%%%%%%%%%%
%	STRAIGHTFORWARD SECTION
%%%%%%%%%%%%%%%%%%%%%%%%%%%%%


\startnewpage

\section*{Section 1: Straightforward questions (50 points)}

\textit{You must show work to get credit. You get two drops: do 5 out
  of the following 7 questions (we will grade all 7 and keep only the
  5 best scores). However, there will be essentially no partial credit
  given in this section. Students who get all 7 questions correct will
  receive some bonus points.}  
%\vspace{-5mm}

\qns{Interpolate (10 points)}  

Prof.~Sahai decides to share his favorite Pok\'emon among six of the
GSIs, but to keep it a secret unless enough GSIs come together. Each
potential favorite Pok\'emon is given a number (Charizard=0,
Wartortle=1, Pidgeot=2, Ninetales=3, Arcanine=4, Scyther=5, Jolteon=6)
and the favorite is hidden as the constant term (i.e.~as usual, the
secret is in $P(0)$) in a polynomial, $P(x)$, of degree $d \leq 4$
using $GF(7)$. You manage to acquire the following five points by
attending different GSI office hours: $(1,0), (2,6), (3,0), (4,0),
(6,0)$. {\bf Use Lagrange Interpolation to find out the secret} (the
Prof's favorite Pok\'emon).

\answer{ 
\[ P(x) = 6 \cdot \frac{(x-1)(x-3)(x-4)(x-6)}{(2-1)(2-3)(2-4)(2-6)} \]
\[ \Rightarrow P(0) = -3 \cdot 3 \cdot 6 = -54 = \boxed{2 \bmod 7} \]
}

\vspace{1in}

\qns{Compute (10 points)} 

Find $300^{300} \bmod 35$.

\answer{
The extension of Fermat's Little Theorem says that if $p$ and $q$ are two distinct primes both bigger than 2, then for any $a$ we have 
\[
a^{(p-1)(q-1)} \equiv 1$ mod $pq
\]
In our case $p=5$ and $q=7$, so we have that $a^{24} = 1$ mod 35 for any $a$. Therefore
\begin{align*}
300^{300} &= (300^{24})^{12} \cdot 300^{12} \\
&= 1^{12} \cdot 20^{12} \\
&= (400)^6 \\
&= 15^6 \\
&= (225)^3 \\
&=15^3 \\
&=225\cdot 15 \\
&= 15 \cdot 15 \\
&= \boxed{15 \bmod 35}
\end{align*}
}
%\vspace{3in}

%\startnewpage

%[Extra page. If you want the work on this page to be graded, make sure you tell us on the problem's main page.]

\startnewpage

\qns{Argue (10 points)} 

In RSA, if Alice wants to send a confidential message to
Bob, she uses Bob's public key to encode it. Then Bob uses his
private key to decode the message. 

Suppose that Bob chose $N = 77$. 

And then Bob chose $e=3$ so his public key is $(3,77)$.

And then Bob chose $d=26$ so his private key is $(26,77)$. 

{\bf Will this work for encoding and decoding messages? If not, where did
Bob \underline{first} go wrong in the above sequence of steps and what is the
consequence of that error. If it does work, then show that it works.}

\answer{We know that $e$ and $d$ should be inverses mod $(p-1)(q-1)$, where $pq=N$. So:
\[ e\cdot d \equiv 1 \mod (p-1)(q-1) \]

In this case, $p=7$ and $q=11$, so $(p-1)(q-1) = 6 \cdot 10 = 60$. However:
\[ e \cdot d = 3 \cdot 26 = 78 = 18 \bmod 60 \]

So $e$ and $d$ are {\em not} inverses mod 60, so this scheme will not work. Bob probably computed $d$ as the inverse of $e$ mod $pq=77$, which is not what we want for RSA.
}
\vspace{1in}

\qns{Prove (10 points)} 

{\bf Prove the following statement:}

If two degree $d \leq n-1$ polynomials $P(x)$ and $Q(x)$ agree at $n$
distinct $x$s (i.e.~For $x_1, x_2, \ldots, x_n$ distinct,
$P(x_i)=Q(x_i)$), then they are the same function everywhere else too.  

\answer{ Let $D(x) = P(x) - Q(x)$.  The degree of $D(x)$ is at most $n-1$. However, we know that $D(x)$ has $n$ distinct roots.  But a non-zero degree $n-1$ polynomial can have at most $n-1$ distinct roots.  Therefore $D(x)$ must be zero everywhere, so $P(x)$ and $Q(x)$ must be the same function everywhere.
}
\vspace{3in}

%\startnewpage

%[Extra page. If you want the work on this page to be graded, make sure you tell us on the problem's main page.]

\startnewpage

\qns{Recognize (10 points)} 

\begin{itemize}

\item[a. ]{I have a large number of biased coins that tend to come up heads $80\%$ of the time, and tails
the other $20\%$. A trial consists of my flipping $k$ such coins, and an experiment
consists of $1000$ such trials. I do $3$ such experiments, with $k$ being $100$, $1000$, and $10000$ respectively. 
For each experiment, I plot a histogram of the number of heads in each
trial, minus $0.8 k$. 
My horizontal axis is the same for all $3$ experiments. {\bf Which histogram below corresponds to which value of $k$?}
\begin{figure*}[htbp]
	\vspace{-0.5em}
	\centering
	{
		\hspace{0em}\includegraphics[width=0.7\linewidth]{resources/figures/plots-a.png}
	}
	\vspace{-0.5em}
    %\caption{\label{probability-plots}}
    \vspace{-0.5em}
\end{figure*}
}

\item[b. ]{I do the same experiment as above, but now I plot
    histograms of the \textit{fraction} of heads (minus 0.8) from each trial. 
    Now {\bf which histogram below corresponds to which value of $k$?}
\begin{figure*}[htbp]
	\vspace{-0.5em}
	\centering
	{
		\hspace{0em}\includegraphics[width=0.70\linewidth]{resources/figures/plots-b.png}
	}
	\vspace{-0.5em}
    %\caption{\label{probability-plots}}
    \vspace{-0.5em}
\end{figure*}
}

\item[c. ]{My friend gives me $3$ bags full of biased coins: coins in these bags come up 
        heads $40\%$, $50\%$, and $60\%$ of the time respectively. Suppose for each 
        $q$ in $0 \leq q \leq 1$, I record the fraction $f(q)$ of
        trials in which I get at most $q$ fraction of heads when I
        flip $100$ coins drawn from one of these bags.  If I plot
        $q$ (x axis) vs $f(q)$ (y axis), {\bf which curve below corresponds
        to which bag? }
\begin{figure*}[htbp]
	\vspace{-0.2em}
	\centering
	{
		\hspace{0em}\includegraphics[width=0.35\linewidth]{resources/figures/plots-c-v2-gimped.png}
	}
	\vspace{-0.5em}
    %\caption{\label{probability-plots}}
    \vspace{-0.5em}
\end{figure*}
}
\end{itemize}

%\vspace{3in}

\startnewpage
\qns{Derive (10 points)} 

You would like to send a message of length $n>0$ over a lossy
channel that drops ({\bf erases}) packets. If up to a fraction
$\frac{1}{4}$ of the {\em total} number of packets you send get {\bf
  erased, how many extra packets do you need to send (as a function of
  $n$)?} 

\answer{
\[ k \geq \frac{1}{4}(n+k) \quad \Rightarrow \quad \boxed{k \geq \frac{n}{3}} \]
}
\vspace{1in}

\qns{Solve \ldots (10 points)}

Alice wants to send Bob a message of length 2 in $GF(7)$ over a noisy channel.
She knows that at most 1 character will get corrupted when she sends
her message. So, she pads her message with 2 extra characters before
sending it. (Using standard interpolation-based $0$-indexed Reed
Solomon codes.)

This is what Bob receives: \textbf{A A E G}

{\bf What was Alice trying to tell him?}

Note: Here, assume that letters correspond to numbers as follows:

$A=0$ \\
$B=1$ \\
$C=2$ \\
$D=3$ \\
$E=4$ \\
$F=5$ \\
$G=6$

%ABCDEFG
%0123456

%\startnewpage

%[Extra page. If you want the work on this page to be graded, make sure you tell us on the problem's main page.]

%%%
\answer{ The four points are (0,0), (1,0), (2,4), (3,6). Let $Q(x) = ax^2+bx+c$, and $E(x) = x-e$. This gives the four equations:
\begin{align*}
c & = 0(0-e) \quad \Rightarrow \quad c=0 \\
a+b & = 0(1-e) \quad \Rightarrow \quad b=-a \\
4a-2a & = 4(2-e) \quad \Rightarrow \quad 2a = 8-4e \\
9a-3a & = 6(3-e) \quad \Rightarrow \quad 6a = 18-6e
\end{align*}
You get $a = 4$, $b = -4$ and $e = 1$.
This gives $Q(x) = 4x^2 - 4x$ and $E(x) = (x-1)$.
The actual polynomial is $P(x) = 4x$, plugging in $x=1$, we get that the message was \boxed{\text{AE}}.
}
%%%


%%%%%%%%%%%%%%%%%%%%%%%%%%%%%
%	TRUE/FALSE SECTION
%%%%%%%%%%%%%%%%%%%%%%%%%%%%%


\startnewpage

\section*{Section 2: True/False (30 points)}

\textit{For the questions in this section, determine whether the
  statement is true or false. If true, prove the statement is true. If 
  false, provide a counterexample demonstrating that it is false.}

%\textit{You get one drop: do 3 out of the following 4 questions.}


\qns{Sums (15 points)}

Suppose that $n \geq 1$ is a positive integer, $x_1,x_2,\ldots,x_n$
are also nonzero positive integers, and $p$ is a prime. Then
\begin{displaymath}
	(x_1+x_2+ \cdots +x_n)^p=x_1^p+x_2^p+ \cdots +x_n^p \pmod{p}.
\end{displaymath}

Mark one: {\sc TRUE} or {\sc FALSE}.

\startnewpage

\qns{Distance (15 points)} 

Let $n \geq 1$ be a positive integer. Let $r \geq 1$ be
a positive integer. Consider a polynomial based code in which
$n$ character messages are encoded into polynomials of degree less
than or equal to $n-1$. The codewords are generated by evaluating
these polynomials at $n+r$ distinct points (assume the underlying
finite field has more than $n+r$ elements). 

Then any two codewords corresponding to different messages must
differ in at least $r+2$ places. 

Mark one: {\sc TRUE} or {\sc FALSE}.

\vspace{3in}

%\startnewpage


%[Extra page. If you want the work on this page to be graded, make sure you tell us on the problem's main page.]

\startnewpage


%%%%%%%%%%%%%%%%%%%%%%%%%%%%%
%	HARDER PROBLEMS SECTION
%%%%%%%%%%%%%%%%%%%%%%%%%%%%%


\section*{Section 3: Free-form Problems (45 points)}

%\textit{You get one drop: do 3 out of the following 4 questions (we
%  will grade all 4 and keep only the 3 best scores). Students who do
%  better than getting all three perfectly will receive some bonus
%  points.}  

\qns{You knew this was coming\ldots (20 points)}

As you know from homework, we can mod polynomials themselves. For this
problem, consider formal polynomials (i.e.~a degree at most $d$ formal
polynomial is something that can be written $\sum_{i=0}^{d} a_i x^i$)
with coefficients in $GF(2)$ (i.e.~the $a_i$ are 0 or 1 with usual binary
math). 

(So, for example, $x^2 + x$ is the remainder of
poly-long-dividing $x^4$ by $x^3 + x + 1$. Meanwhile, the quotient of
the same division is just $x$. This is because $x^4 = x(x^3 + x + 1) +
(x^2 + x)$ when all arithmetic on coefficients is performed mod 2. )

{\bf Compute the multiplicative inverse of the formal polynomial $x+1$ in mod $x^3
+ x + 1$.} That is, give a formal polynomial $P(x)$ so that $P(x)(x+1)
\bmod (x^3 + x + 1) = 1$. 

{\em (e.g.~The multiplicative inverse of $x$ is $x^2 + 1$ because
  $x(x^2 + 1) \bmod (x^3 + x + 1) = 1$.)}

\answer{
We can use egcd to find the multiplicative inverse of $x+1$ mod $x^3+x+1$.  First we divide $x^3+x+1$ by $x+1$, and we get a remainder of 1.
\begin{align*}
& \text{egcd}(x+1,x^3+x+1) \\
= & \text{egcd}(x+1,1) & 1 = (x^3+x+1) - (x^2+x)(x+1) 
\end{align*}

Therefore the inverse of $(x+1)$ is $-(x^2+x)$. But because we're working mod 2, this is the same as $\boxed{x^2+x}$.
}

%\startnewpage

%[Extra page. If you want the work on this page to be graded, make sure you tell us on the problem's main page.]

\startnewpage

\qns{Magic Command (25 points)}

You are a young technomage (one who uses technology to create the
impression of magic) who has been asked to help some people on a
planet with a fragile new peace treaty. They have a master computer that
is connected to a doomsday device capable of blowing up the planet if
given the publicly-known command ``Magic computer, please blow up the
world now.'' (For the purposes of this problem, feel free to think of
this as being a publicly-known magic number like 42). 

% unleash a terrible
% storm if given the command ``superstorm'' that could hurt lots of
% people. It can also be used for helpful tasks like ``please make a
% sunny day'' and ``we'd like a gentle rain of 2 inches spread over 4
% days.'' 

The problem is that the computer has no security on it. It just
accepts plain text commands. 

\begin{itemize}
%\qpart
\item[a. (10 pts)] You have been asked to add some
security to the system to prevent unauthorized use. You can remove the
keyboard, add a tamper-proof-decryption module, and force all keyboard
input to go through the decryption module before it goes to the
computer. But anyone can walk up to the decryption module to study it
because it is in the {\bf public} square. {\bf Please show how you
  would design such a public module that transforms inputs before
  feeding them into the computer.} This module should effectively make
the magic number that blows up the world into something secret. 

You can use modulo math operations as you see fit. (You don't have to
reprove anything you have seen in lecture or notes.)

\startnewpage
%\qpart
\item[b. (15 pts)] The people on this planet are divided into two
  factions: the blues and the golds. Within each color group, there
  are 4 families. They have agreed on the following behavior: 

If at least 2 blue families and at least 2 gold families come
together, they should be able to give a valid command to the master
computer. In addition, if all the blues agree or all the golds agree,
then they should also be able to give a valid command to the master
computer. But no other grouping should be able to do it. (e.g.~3 blues
and 1 gold {\em should not} be able to give a command.)

{\bf Design a scheme and argue why it works as intended.} You can use
modulo  math operations as you see fit, as well as give pieces of
information {\bf secretly} to families. (You don't have to reprove
anything you have seen in lecture or notes.)
\end{itemize}

\startnewpage

[Extra page. If you want the work on this page to be graded, make sure you tell us on the problem's main page.]

\startnewpage

\qns{(Optional) Multiplication (20 points)}

Suppose you have invented a machine for doing \emph{mod multiplications} 
extremely fast. 

Given a prime $p$, your invention takes two equal size lists of
numbers from $\{0,1,\ldots, p-1\}$ as inputs and returns the
element-wise product of the input lists, mod $p$. Your machine is fast
and can accomodate any size lists, but it is also prone to
mistakes. More specifically, you know that at most $\frac{1}{3}$ of
the results are wrong. For example, suppose $p=29$ and we feed the
machine the two lists $(1, 2, 6)$ and $(5, 1, 2)$ we might get back
output $(5, 2, 11)$ where $11 \neq 6 \times 2$ is a mistake. None of
your potential clients is interested in a device which returns false
results. 

{\bf Show that you can augment your machine with Reed-Solomon-like
encoding and decoding schemes such that no wrong outputs are ever
returned and correct answers are obtained to the $m$ pairs of numbers
that you actually want to multiply together.} (You will need to ask the
machine itself to multiply more than $m$ pairs of numbers to
accomplish this.) You can assume you have access to
interpolation-based ($1$-indexed) RS-encoders (parameters like $n$ and
$k$ can be adjusted as you would like) as well as Berlekamp-Welch
decoders. (Again, internal parameters like $n$ and $k$ can be
{\bf independently} adjusted as you would like --- in particular, you
don't have to use the same $n$ and $k$ that you used for the encoders.)

For concreteness, you can assume that the size of the desired input
lists is $m=6$ and that $p=127$. (This is a prime number)

{\em (HINT: first explore what happens with $m=1$ and $m=2$ to get an
  idea for what is going on.)}

\startnewpage

[Extra page. If you want the work on this page to be graded, make sure you tell us on the problem's main page.]


\startnewpage

%[Extra page. If you want the work on this page to be graded, make sure you tell us on the problem's main page.]

[Doodle page! Draw us something if you want or give us suggestions or complaints.]


\end{qunlist}


\end{document}

\qns{Indexing Dispute (20 points)}

The EECS 70 Piazza group has blown up around this
Reed-Solomon puzzle:

Eve and Dan both use Reed-Solomon codes where the encoding is based
on interpolation and the message symbols are at the beginning of
codewords. But Eve likes to use $0$-indexing while Dan likes to use
$1$-indexing. For a prime $p$, Eve's codewords are $(P(0), P(1), P(2),
\ldots, P(p-1))$ in that order. Dan's codewords are instead $(P(1),
P(2), P(3), \ldots, P(p-1), P(0))$.

Suppose $n+2k = p$. When Eve uses her encoder to protect $n$ symbols,
she can tolerate up to $k$ general errors in the received symbols
since when she feeds them into her own Berlekamp-Welch decoding box,
that box will reliably emit the original $n$ symbols if $k$ or fewer
symbols were corrupted. The same for Dan's encoder and decoder when
used together. 

The question is what happens if we use Eve's encoder together with
Dan's decoder. Does this still work? If not why not. If so, how well
and why? 

{\em (HINT: first explore what happens with $n=1$ even with errors and 
  $n=2$ with no errors to get an idea for what is going on.)}

\startnewpage

[Extra page. If you want the work on this page to be graded, make sure you tell us on the problem's main page.]

\startnewpage


% LocalWords:  maketitle noindent sc cs70 hline Ramtin Chenyu Sibi Dwin hskip
% LocalWords:  framebox newpage tabularnewline qquad vspace mansplains geq2 qns
% LocalWords:  Rightarrow Rightarrow vfill mbox cdot cdot bmod bmod forall fbox
% LocalWords:  mathbb charbox charbox charbox charbox charbox charbox charbox
% LocalWords:  charbox charbox qunlist 2pts pageref startnewpage approx sqrt nx
% LocalWords:  hspace leq hspace infty includegraphics textwidth textit emon
% LocalWords:  Wartortle Pidgeot Ninetales Arcanine Scyther Jolteon
