\documentclass[11pt]{article}
\usepackage{../../cs70}
\usepackage{color}
\newif\ifsolutions
\solutionstrue
\solutionsfalse %flag for solutions

\def\title{Discussion 7B}
\begin{document}
\maketitle

\begin{enumerate}

% 
\item Find (and prove) an upper-bound on the number of times two distinct degree $d$ polynomials can intersect. What if the polynomials' degrees differ?

\ifsolutions
\textbf{Motivation for Problem:} 

\textbf{Solutions:} 
\fi

%\vspace{30mm}

% previous #5
\item Suppose your GSI has chosen another polynomial, $Q(x)$, of degree $\leq 2$, 
and told the value of $Q(1)$ to their favorite student, $Q(2)$ to their second favorite student, and so on, 
up through $Q(11)$ this time. (All of this is modulo $11$, as in last week's discussion.)
\begin{enumerate}
\item Alas, you are not among your GSI's favorite eleven students. How many of those students would you need to talk to before 
you could figure out the coefficients of Q(x)?

\item That's assuming the students are honest. But, it turns out, four of your GSI's favorite students are morally 
bankrupt and perfectly willing to lie to you. (The rest are, thankfully, completely trustworthy.)

If you talk to $7$ students, what is the minimum number of correct answers you will receive?
\item If you talk to $7$ students, how can you tell whether at least one of them is lying? {\em Hint: You may find it useful to think back to the first question.}
\item Is there a group of $7$ students with no liars?
\item With access to all $11$ students, how can you figure out the polynomial $Q(x)$, 
despite the fact that $4$ students are untrustworthy? [Do not worry how efficient your method is]
\item If your GSI's polynomial had been of degree up to $4$, instead, how many untrustworthy students could there be before 
this method would stop working? What should all the 7s above be changed to in that case?
\end{enumerate}


\ifsolutions 
\textbf{Solutions:} \fi

%\vspace{30mm}


\item In this problem we will see how to send 4 bits (each either 0 or 1) using Hamming codes. This will be accomplished by using 3 parity bits (check bits) so that 1 error in the message can be detected and corrected. To encode a 4-bit message $p$, we will multiply it by a code-generator matrix $G$:
\[ G = \left( \begin{array}{cccc} 1 & 1 & 0 & 1 \\
1 & 0 & 1 & 1 \\
1 & 0 & 0 & 0 \\
0 & 1 & 1 & 1 \\
0 & 1 & 0 & 0 \\
0 & 0 & 1 & 0 \\
0 & 0 & 0 & 1 \\
\end{array} \right)
\]   

\begin{enumerate}

\item Show how to encode the message $p=1101$.  What is the resulting 7-bit transmitted message $x$?
\item Suppose that there was an error (flipped bit) in the 6th bit of $x$. To find out where it occurred, we can multiply $x$ by a parity check matrix $H$:
\[ H = \left( \begin{array}{ccccccc} 1 & 0 & 1 & 0 & 1 & 0 & 1 \\
0 & 1 & 1 & 0 & 0 & 1 & 1 \\
0 & 0 & 0 & 1 & 1 & 1 & 1
\end{array} \right)
\]   

What is the resulting binary number?  What is this number base 10? How can you correct $x$?

\item Now we wish to recover the original message $p$. To do this, we can multiply the corrected $x$ by a recovery matrix $R$ that pulls out the 4 original bits:

\[ R = \left( \begin{array}{ccccccc} 0 & 0 & 1 & 0 & 0 & 0 & 0 \\
0 & 0 & 0 & 0 & 1 & 0 & 0 \\
0 & 0 & 0 & 0 & 0 & 1 & 0 \\
0 & 0 & 0 & 0 & 0 & 0 & 1
\end{array} \right)
\]   

Do you get $p$ back?
\end{enumerate}
 \end{enumerate}




\end{document}
