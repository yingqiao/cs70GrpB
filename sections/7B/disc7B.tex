\documentclass[11pt]{article}
\usepackage{../../cs70}
\usepackage{color}
\newif\ifsolutions
%\solutionstrue
\solutionsfalse %flag for solutions

\def\title{Discussion 7B}
\begin{document}
\maketitle

\begin{enumerate}

% 
\item {\bf (Polynomial intersections)} Find (and prove) an upper-bound on the number of times two distinct degree $d$ polynomials can intersect. What if the polynomials' degrees differ?

\vspace{20mm}

\ifsolutions
%\textbf{Motivation for Problem:} 
\vspace{-20mm}
{\color{blue}{
\textbf{Solutions:} They can intersect up to $d$ times. This is because we can specify $d$ points to be the same for both polynomials, then the $(d+1)$th point to be different.  Then the polynomials will be distinct, but still agree at $d$ points.  If $d+1$ points agree, the polynomials will be identical.

If the polynomials have degrees $d_1$ and $d_2$, with $d_1 > d_2$, then they can intersect up to $d_1$ times. Pick $d_1$ points on the polynomial of degree $d_2$, and another point not on this polynomial. Then a unique degree $d_1$ polynomial will go through these $d_1+1$ points.}}
\fi

%\vspace{30mm}

% previous #5
\item {\bf (Berlekamp-Welch for general errors)} Suppose that Hector wants to send you a length $n=3$ message, $m_0,m_1,m_2$, with the possibility for $k=1$ error. In this world we will work mod 11, so we can encode 11 letters as shown below:
\[ \begin{tabular}{|ccccccccccc|}
\hline
A & B & C & D & E & F & G & H & I & J & K \\
\hline
0 & 1 & 2 & 3 & 4 & 5 & 6 & 7& 8 & 9 & 10 \\
\hline
\end{tabular} \]

Hector encodes the message by finding the degree $\leq 2$ polynomial $P(x)$ that passes through $(0,m_0)$, $(1,m_1)$, and $(2,m_2)$. He then also sends $(3,m_3)$ and $(4,m_4)$. The message you receive is 
\[ \text{DHACK} \Rightarrow 3,7,0,2,10 = r_0,r_1,r_2,r_3,r_4 \]
which could have up to 1 error.

\begin{enumerate}
\item First locate the error, using an error-locating polynomial $E(x)$.  Let $Q(x) = P(x)E(x)$. Recall that
\[ Q(i) = P(i)E(i) = r_i E(i), \quad \text{for} \quad 0 \leq i < n+2k \]

What is the degree of $E(x)$? What is the degree of $Q(x)$? Using the relation above, write out the form of $E(x)$ and $Q(x)$, and then a system of equations to find both these polynomials.
\vspace{20mm}

\ifsolutions
\vspace{-20mm}
{\color{blue}{
\textbf{Solutions:} The degree of $E(x)$ will be 1, since there is at most 1 error. The degree of $Q(x)$ will be 3, since $P(x)$ is of degree 2. $E(x)$ will have the form $E(x) = x + e$, and $Q(x)$ will have the form $Q(x) = ax^3 + bx^2 + cx + d$. We can write out a system of equations to solve for these 5 variables:
\begin{align*}
d & = 3(0+e) \\
a+b+c+d & = 7(1+e) \\
8a+4b+2c+d & = 0(2+e) \\
27a+9b+3c+d & = 2(3+e) \\
64a+16b+4c+d & = 10(4+e)
\end{align*}
}}
\fi

\item Ask your GSI for $Q(x)$. What is $E(x)$? Where is the error located?
\vspace{20mm}

\ifsolutions
\vspace{-20mm}
{\color{blue}{
\textbf{Solutions:} Solving this system (with Wolfram Alpha) we get 
\[ Q(x) = 3x^3 + 6x^2 + 5x + 8\]
Plugging into the first equation (for example), we see that:
\[ d = 8 = 3e \quad \Rightarrow \quad e = 8 \cdot 4 = 32 \equiv 10 \bmod 11 \]

This means that 
\[ E(x) = x+10 \equiv x - 1 \mod 11. \]

Therefore the error occurred when $x=1$ (so the second number sent in this case).
}}
\fi

\item Finally, what is $P(x)$? Use $P(x)$ to determine the original message that Hector wanted to send.

\ifsolutions
%\vspace{-20mm}
{\color{blue}{
\textbf{Solutions:} Using polynomial division, we divide $Q(x) = 3x^3 + 6x^2 + 5x + 8$ by $E(x) = x-1$:
\[ P(x) = 3x^2 + 9x + 3 \]

Then $P(1) = 3+9+3 = 15 \equiv 4 \bmod 11$. This means that our original message was
\[ 3,4,0 \quad \Rightarrow \quad \text{DEA} \]
}}
\fi

\end{enumerate}

\newpage

%Suppose your GSI has chosen another polynomial, $Q(x)$, of degree $\leq 2$, 
%and told the value of $Q(1)$ to their favorite student, $Q(2)$ to their second favorite student, and so on, 
%up through $Q(11)$ this time. (All of this is modulo $11$, as in last week's discussion.)
%\begin{enumerate}
%\item Alas, you are not among your GSI's favorite eleven students. How many of those students would you need to talk to before 
%you could figure out the coefficients of Q(x)?
%
%\item That's assuming the students are honest. But, it turns out, four of your GSI's favorite students are morally 
%bankrupt and perfectly willing to lie to you. (The rest are, thankfully, completely trustworthy.)
%
%If you talk to $7$ students, what is the minimum number of correct answers you will receive?
%\item If you talk to $7$ students, how can you tell whether at least one of them is lying? {\em Hint: You may find it useful to think back to the first question.}
%\item Is there a group of $7$ students with no liars?
%\item With access to all $11$ students, how can you figure out the polynomial $Q(x)$, 
%despite the fact that $4$ students are untrustworthy? [Do not worry how efficient your method is]
%\item If your GSI's polynomial had been of degree up to $4$, instead, how many untrustworthy students could there be before 
%this method would stop working? What should all the 7s above be changed to in that case?
%\end{enumerate}

%\vspace{30mm}


\item {\bf (Hamming codes)} Here is an example of a simple single-error correcting code that works without having to invoke polynomials and just deals with bits. In real life, these kinds of codes are often used in conjunction with Reed Solomon codes. In this problem we will see how to send 4 bits (each either 0 or 1) using Hamming codes. This will be accomplished by using 3 parity bits (check bits) so that 1 error in the message can be detected and corrected. To encode a 4-bit message $m$, we will multiply it by a code-generator matrix $G$:
\[ G = \left( \begin{array}{cccc} 1 & 1 & 0 & 1 \\
1 & 0 & 1 & 1 \\
1 & 0 & 0 & 0 \\
0 & 1 & 1 & 1 \\
0 & 1 & 0 & 0 \\
0 & 0 & 1 & 0 \\
0 & 0 & 0 & 1 \\
\end{array} \right)
\]   

\begin{enumerate}

\item Show how to encode the message $m=1101$.  What is the resulting 7-bit transmitted message $x$?
\vspace{15mm}

\ifsolutions 
\vspace{-15mm}
{\color{blue}{
\textbf{Solutions:} $Gm = x = 1010101$.
}}
\fi

\item To detect potential errors, we can use a parity check matrix $H$:
\[ H = \left( \begin{array}{ccccccc} 1 & 0 & 1 & 0 & 1 & 0 & 1 \\
0 & 1 & 1 & 0 & 0 & 1 & 1 \\
0 & 0 & 0 & 1 & 1 & 1 & 1
\end{array} \right)
\]

What do you notice about the form of $H$?  If you multiply $H$ by $x$, what do you get? Will this be true for any message $m$?
\vspace{15mm}

\item Suppose that there was an error (flipped bit) in the 6th bit of $x$, yielding $x'$. To find out where it occurred, we can multiply $H$ by $x'$. What is the result base 10? How can you correct $x$?
\vspace{15mm}

\ifsolutions 
\vspace{-15mm}
{\color{blue}{
\textbf{Solutions:} We actually received $x' = 1010111$. Computing $Hx'$, we get the sixth column of $H$, which corresponds to 6 in binary, so we know the 6th bit should be flipped to correct $x$.
}}
\fi

\item Now we wish to recover the original message $p$. To do this, we can multiply the corrected $x$ by a recovery matrix $R$ that pulls out the 4 original bits:

\[ R = \left( \begin{array}{ccccccc} 0 & 0 & 1 & 0 & 0 & 0 & 0 \\
0 & 0 & 0 & 0 & 1 & 0 & 0 \\
0 & 0 & 0 & 0 & 0 & 1 & 0 \\
0 & 0 & 0 & 0 & 0 & 0 & 1
\end{array} \right)
\]   

Do you get $m$ back?

\vspace{15mm}

\ifsolutions 
%\vspace{-20mm}
{\color{blue}{
\textbf{Solutions:} After correcting $x$, we compute $Rx = m$, and get back $m=1101$.
}}
\fi

\item Do you think that a code like this could exist that could fix a single binary error in a 4-bit message without using at least 7 bits for the codeword? Why or why not?

\end{enumerate}
 \end{enumerate}




\end{document}
