\documentclass[11pt]{article}
\usepackage{../../cs70}
\usepackage{color}
\newif\ifsolutions
\solutionstrue
\solutionsfalse %flag for solutions

\def\title{Discussion 7B}
\begin{document}
\maketitle

\begin{enumerate}

% 
\item {\bf (Polynomial intersections)} Find (and prove) an upper-bound on the number of times two distinct degree $d$ polynomials can intersect. What if the polynomials' degrees differ?

\vspace{20mm}

\ifsolutions
\textbf{Motivation for Problem:} 

\textbf{Solutions:} 
\fi

%\vspace{30mm}

% previous #5
\item {\bf (Berlekamp-Welch for general errors)} Suppose that Hector wants to send you a length $n=3$ message, $m_0,m_1,m_2$, with the possibility for $k=1$ error. In this world we will work mod 11, so we can encode 11 letters as shown below:
\[ \begin{tabular}{|ccccccccccc|}
\hline
A & B & C & D & E & F & G & H & I & J & K \\
\hline
0 & 1 & 2 & 3 & 4 & 5 & 6 & 7& 8 & 9 & 10 \\
\hline
\end{tabular} \]

Hector encodes the message by finding the degree $\leq 2$ polynomial $P(x)$ that passes through $(0,m_0)$, $(1,m_1)$, and $(2,m_2)$. He then also sends $(3,m_3)$ and $(4,m_4)$. The message you receive is 
\[ \text{DHACK} \Rightarrow 3,7,0,2,10 = r_0,r_1,r_2,r_3,r_4 \]
which could have up to 1 error.

\begin{enumerate}
\item First locate the error, using an error-locating polynomial $E(x)$.  Let $Q(x) = P(x)E(x)$. Recall that
\[ Q(i) = P(i)E(i) = r_i E(i), \quad \text{for} \quad 0 \leq i < n+2k \]

What is the degree of $E(x)$? What is the degree of $Q(x)$? Using the relation above, write out the form of $E(x)$ and $Q(x)$, and then a system of equations to find both these polynomials.
\vspace{20mm}

\item Ask your GSI for $Q(x)$. What is $E(x)$? Where is the error located?
%It turns out that $Q(x) = 3x^3 + 6x^2 + 5x + 8$. 
\vspace{20mm}

\item Finally, what is $P(x)$? Use $P(x)$ to determine the original message that Hector wanted to send.
\newpage
\end{enumerate}

%Suppose your GSI has chosen another polynomial, $Q(x)$, of degree $\leq 2$, 
%and told the value of $Q(1)$ to their favorite student, $Q(2)$ to their second favorite student, and so on, 
%up through $Q(11)$ this time. (All of this is modulo $11$, as in last week's discussion.)
%\begin{enumerate}
%\item Alas, you are not among your GSI's favorite eleven students. How many of those students would you need to talk to before 
%you could figure out the coefficients of Q(x)?
%
%\item That's assuming the students are honest. But, it turns out, four of your GSI's favorite students are morally 
%bankrupt and perfectly willing to lie to you. (The rest are, thankfully, completely trustworthy.)
%
%If you talk to $7$ students, what is the minimum number of correct answers you will receive?
%\item If you talk to $7$ students, how can you tell whether at least one of them is lying? {\em Hint: You may find it useful to think back to the first question.}
%\item Is there a group of $7$ students with no liars?
%\item With access to all $11$ students, how can you figure out the polynomial $Q(x)$, 
%despite the fact that $4$ students are untrustworthy? [Do not worry how efficient your method is]
%\item If your GSI's polynomial had been of degree up to $4$, instead, how many untrustworthy students could there be before 
%this method would stop working? What should all the 7s above be changed to in that case?
%\end{enumerate}


\ifsolutions 
\textbf{Solutions:} \fi

%\vspace{30mm}


\item {\bf (Hamming codes)} In this problem we will see how to send 4 bits (each either 0 or 1) using Hamming codes. This will be accomplished by using 3 parity bits (check bits) so that 1 error in the message can be detected and corrected. To encode a 4-bit message $p$, we will multiply it by a code-generator matrix $G$:
\[ G = \left( \begin{array}{cccc} 1 & 1 & 0 & 1 \\
1 & 0 & 1 & 1 \\
1 & 0 & 0 & 0 \\
0 & 1 & 1 & 1 \\
0 & 1 & 0 & 0 \\
0 & 0 & 1 & 0 \\
0 & 0 & 0 & 1 \\
\end{array} \right)
\]   

\begin{enumerate}

\item Show how to encode the message $p=1101$.  What is the resulting 7-bit transmitted message $x$?
\vspace{30mm}
\item Suppose that there was an error (flipped bit) in the 6th bit of $x$. To find out where it occurred, we can multiply $x$ by a parity check matrix $H$:
\[ H = \left( \begin{array}{ccccccc} 1 & 0 & 1 & 0 & 1 & 0 & 1 \\
0 & 1 & 1 & 0 & 0 & 1 & 1 \\
0 & 0 & 0 & 1 & 1 & 1 & 1
\end{array} \right)
\]   

What is the resulting binary number?  What is this number base 10? How can you correct $x$?
\vspace{30mm}

\item Now we wish to recover the original message $p$. To do this, we can multiply the corrected $x$ by a recovery matrix $R$ that pulls out the 4 original bits:

\[ R = \left( \begin{array}{ccccccc} 0 & 0 & 1 & 0 & 0 & 0 & 0 \\
0 & 0 & 0 & 0 & 1 & 0 & 0 \\
0 & 0 & 0 & 0 & 0 & 1 & 0 \\
0 & 0 & 0 & 0 & 0 & 0 & 1
\end{array} \right)
\]   

Do you get $p$ back?
\end{enumerate}
 \end{enumerate}




\end{document}
