\documentclass[11pt]{article}
\usepackage{../../cs70}
\usepackage{color}
\newif\ifsolutions
\solutionstrue
%\solutionsfalse %flag for solutions

\def\title{Discussion 13B}
\begin{document}
\maketitle


%-polling questions, ideally something that ends up with having to solve a quadratic equation (he suggested something similar to a problem in note 19)
%
%-coupon collecting, instead of expectation, compute the variance
%
%-error-correcting code where you compute the channel error, given that you want the overall error to be less than some value
%
%-I think the goal is that all of these will end up using Chebyshev's inequality.

\begin{enumerate}

\item {\bf Warmup}
\begin{enumerate}[a)]
\item Calculate the Variance of a $X$ where $X$ represents the value of a standard 6-sided dice.
\item Calculate the variance of the binomial distribution
\end{enumerate}

\vspace{20mm}

\item {\bf Homework Polling - Revisited} 

Suppose Prof. Sahai wants to poll the CS 70 students {\em again} about whether the homeworks have been too hard recently.  But now everyone is comfortable enough to answer honestly, either no (0) or yes (1). Let the true fraction of students who think the homework is too hard be $p$. Let the response of the $i^{\text{th}}$ polled student be $X_i$.  Prof. Sahai would like to poll $n$ students, with $n$ large enough that the probability of estimating $p$ to within 2\% is at least 90\%.

\begin{enumerate}[a)]

\item Let $M_n = \frac{1}{n}\sum_{i=1}^n X_i$ be the average of the $n$ responses. What is the expectation of $M_n$? Write an equation that describes the desired probability. Draw a picture of the distribution of $M_n - p$ and mark the region where the desired probability would be obtained.

\vspace{20mm}

\end{enumerate}

\item {\bf How Many Coupons}
Recall the coupon collecting problem covered in note 17. There are $n$ distinct coupons that wish to collect all $n$ different types of coupons. Every time we buy a box, there is one coupon in, with equal likelihood of being any one of the types of coupons. We want to figure out how many boxes do we need to buy to get one of each coupon. For this problem, we want to bound the probability we have to buy more than $n\log{n}$ coupons.
\begin{enumerate}[a)]
\item We represent $X$, the number of boxes we have to buy, as a sum of other random variables. Let $X_i$ represent the number of boxes you buy to go from $i-1$ to $i$ distinct coupons. Write $X$ as a sum of $X_i$'s. Argue that each $X_i$ is an independent geometric distribution.

\ifsolutions{\color{blue}{
$X = \sum\nolimits_{i=0}^n X_i$ \hspace{5mm} Each $X_i \sim Geo(\frac{n-i}{n})$
}}\fi

\item We wish to use Chebyshev's inequality to bound the probability we have to buy more than $n \log{n}$ boxes. In order to do this, we need to compute the variance of a geometric random variable. We will use the following lemma in our proof: $\sum\nolimits_{k=1}^{\infty} k(k+1)(1-p)^{k-1} = \frac{2}{p^3}$. Prove this lemma. (Hint: what is the sum of the geometric series $\sum\nolimits_{k=0}^{\infty} (1-p)^k$. Take the derivative of both sides, what happens?)

\ifsolutions{\color{blue}{
$\sum\nolimits_{k=0}^{\infty} (1-p)^k = \frac{1}{p}$ take derivative of both sides twice. \\ $-\sum\nolimits_{k=0}^{\infty} k(1-p)^{k-1} = -\frac{1}{p^2}$  \\ $\sum\nolimits_{k=0}^{\infty} k(k-1)(1-p)^{k-2} = \frac{2}{p^3}$ Notice that $k=0$ means the LHS is 0
\\ $\sum\nolimits_{k=1}^{\infty} k(k-1)(1-p)^{k-2} = \frac{2}{p^3}$ 

$\sum\nolimits_{j=0}^{\infty} (j+1)j(1-p)^{j-1} = \frac{2}{p^3}$ change summation bounds  $j = k-1$

}}\fi




\item If $X_i \sim Geom(p)$, show that $\mathbb{E}[X_i^2] = \sum\nolimits_{k=1}^{\infty} k(k+1)p(1-p)^{k-1} - \sum\nolimits_{k=1}^{\infty} kp(1-p)^{k-1}$

\ifsolutions{\color{blue}{
By definition $\mathbb{E}[X_i^2] = \sum\nolimits_{k=1}^{\infty} k^2p(1-p)^{k-1}$ since $k^2 = k(k+1) - k$ we plug that in to get the answer 
}}\fi



\item Use your lemma and the fact that $\mathbb{E}[X_i] = 1/p$ to simplify part c to show $\mathbb{E}[X_i^2] = \frac{2}{p^2} - \frac{1}{p} $
\ifsolutions{\color{blue}{
We simply plug in our lemma to the left hand term. We also notice that the right hand expression is the expectation of a geometric variable with parameter $p$.
}}\fi
\item Show that variance of a geometric variable with parameter $p$ is $\frac{1-p}{p^2}$ We will use the estimation, $Var[X_i] < \frac{1}{p^2}$
\ifsolutions{\color{blue}{
$Var[X] = \mathbb{E}[X^2] - \mathbb{E}[X]^2$ \\ $Var[X_i] = \frac{2}{p^2} - \frac{1}{p}  - \frac{1}{p}^2 = 1/p^2 - 1/p = \frac{1-p}{p^2} < \frac{1}{p^2}$
}}\fi
\item Make use of the fact that $\sum\nolimits_{i=1}^{\infty} \frac{1}{i^2}$ is a positive constant $\leq2$ to show that the $Var[X] \leq 2n^2$ 

\ifsolutions{\color{blue}{
We use the fact that since the $X_i$ are independent, \\ $Var[X] = \sum\nolimits_{i=0}^{n-1} Var[X_i]$ \\ $ \leq \sum\nolimits_{i=0}^{n-1} (\frac{n}{n-i})^2 = n^2\sum\nolimits_{i=0}^{n-1} \frac{1}{i}^2 \leq n^2\sum\nolimits_{i=0}^{\infty} \frac{1}{i}^2 \leq 2n^2$
}}\fi
\item Use Chebyshev's inequality to show that $Pr[ X \geq n^{3/2} + n] \leq 2/n $. (Hint: Recall that we estimated $\mathbb{E}[X] \approx \ln{n} + \gamma \ll n$. Note that $Pr[X \geq \alpha] \geq Pr[X \geq \alpha + y]$ for any positive constant, $y$).

\ifsolutions{\color{blue}{
$n^{3/2} + n > n^{3/2} + \mathbb{E}[X]$ so \\ $Pr[ X \geq n^{3/2} + n] \leq Pr[ X \geq n^{3/2} + \mathbb{E}[X]] \\ =  Pr[ X - \mathbb{E}[X] \geq n^{3/2} ] \leq Pr[ | X - \mathbb{E}[X] | \geq n^{3/2} ] \leq \frac{Var[X]}{n^3} \leq \frac{2n^2}{n^3} = 2/n$
}}\fi


\end{enumerate}


\end{enumerate}

\end{document}
