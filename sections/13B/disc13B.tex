\documentclass[11pt]{article}
\usepackage{../../cs70}
\usepackage{color}
\newif\ifsolutions
\solutionstrue
\solutionsfalse %flag for solutions

\def\title{Discussion 13B}
\begin{document}
\maketitle


%-polling questions, ideally something that ends up with having to solve a quadratic equation (he suggested something similar to a problem in note 19)
%
%-coupon collecting, instead of expectation, compute the variance
%
%-error-correcting code where you compute the channel error, given that you want the overall error to be less than some value
%
%-I think the goal is that all of these will end up using Chebyshev's inequality.

\begin{enumerate}

\item {\bf Warmup}
\begin{enumerate}[a)]
\item Calculate the variance of a random variable, $X$, where $X$ represents the value of a standard 6-sided dice.

\ifsolutions{\color{blue}{
\[ \Pr[X=i] = \frac{1}{6} \quad\text{for } i=1,2,\dots,6 \]
\[ E[X] = \sum_{i=1}^6 i\Pr[X=i] = \frac{1}{6}(1+2+3+\dots+6) = \frac{7}{2} \]
\[ E[X^2] = \sum_{i=1}^6 i^2\Pr[X=i] = \frac{1}{6}(1+4+9+\dots+36) = \frac{91}{6} \]
\[ Var[X] = E[X^2] - (E[X])^2 = \frac{35}{12} \]
}}
\fi


\item Calculate the variance of a random variable defined by the binomial distribution.

\ifsolutions{\color{blue}{
From HW12, we know that $X = \sum_{i=1}^n X_i$, where
\begin{equation*}
X_i = \begin{cases}
      1 &\mbox{with probability } p \\
      0 &\mbox{with probability } (1-p)
      \end{cases}
\end{equation*}
and all these $X_i$'s are i.i.d. Bernoulli r.v. So, $Var[X] = \sum_{i=1}^n Var[X_i]$.
\[ E[X_i] = 1\cdot p + 0\cdot (1-p) = p \]
\[ E[X_i^2] = 1^2\cdot p + 0^2\cdot (1-p) = p \]
\[ Var[X_i] = E[X_i^2] - (E[X_i])^2 = p - p^2 = p(1-p) \]
So, $Var[X] = np(1-p)$.
}}
\fi

\end{enumerate}

\vspace{25mm}

\item {\bf Homework Polling - Revisited} 

Suppose Prof. Sahai wants to poll the CS 70 students {\em again} about
whether the homeworks have been too hard recently.  But now everyone
is comfortable enough to answer honestly, either no (0) or yes
(1). Let the true fraction of students who think the homework is too
hard be $p$. Let the response of the $i^{\text{th}}$ polled student be
$X_i$.  Prof. Sahai would like to poll $n$ students, with $n$ large
enough that the probability of estimating $p$ to within $\pm 2\% $ is at least 90\%.

\begin{enumerate}[a)]

\item Let $M_n = \frac{1}{n}\sum_{i=1}^n X_i$ be the average of the
  $n$ responses. What is the expectation of $M_n$?  

  What is the event that we are interested in whose probability we
  would like to be at least $90\%$? Draw a picture of the distribution
  of $M_n - p$ and mark the region that corresponds to the event of
  interest. 

\ifsolutions{\color{blue}{
\begin{equation*}
X_i = \begin{cases}
      1 &\mbox{hw is too hard} \\
      0 &\mbox{hw is not too hard}
      \end{cases}
\end{equation*}
$X_i$ is i.i.d Bernoulli r.v. with $\Pr[X_i=1] = p$. So, $M_n \sim \frac{1}{n}Binom(n,p)$:
\[ E[M_n] = \frac{1}{n}\sum_{i=1}^n E[X_i] = \frac{1}{n}\cdot(np) = p\]
We want 
\[ \Pr[|M_n - p| < 0.02p] > 0.9 \]
Or equivalently,
\[ \Pr[|M_n - p| \geq 0.02p] \leq 0.1 \]
}}\fi


\item Now use Chebyshev's inequality to find a safe $n$ regardless of
  what $p$ is. 
\ifsolutions{\color{blue}{
\[ Var[M_n] = \frac{1}{n^2} \sum_{i=1}^n Var[X_i] = \frac{p(1-p)}{n} \]
\[ \Pr[|M_n - p|] \geq 0.02p] \leq \frac{Var[M_n]}{(0.02p)^2} = \frac{1-p}{(0.02)^2np} \leq 0.1 \]
\[ n \geq 25000\frac{1-p}{p}\]
}}\fi

\item What if instead of wanting an accuracy of $\pm 2\%$ we wanted a
  relative error of 2\%. This means that if the true value was $p$, we
  want the answer we get to be within $[0.98 p, 1.02p]$. Can we pull
  that off using Chebyshev's inequality? Do you think we could pull
  that off with a single universal choice of $n$ that does not depend
  on (the unknown) $p$?


\end{enumerate}

\newpage

\item {\bf How Many Coupons}

Consider the coupon collecting problem covered in note 17. There are
$n$ distinct types of coupons that we wish to collect. Every time we
buy a box, there is one coupon in it, with equal likelihood of being
any one of the types of coupons. We want to figure out how many boxes
we need to buy in order to get one of each coupon. For this problem,
we want to bound the probability that we have to buy lots of coupons
--- say substantially more than $n\log{n}$ coupons. 

\begin{enumerate}[a)]
\item We represent $X$, the number of boxes we have to buy, as a sum
  of other random variables. Let $X_i$ represent the number of boxes
  you buy to go from $i-1$ to $i$ distinct coupons in your hand. Write
  $X$ as a sum of $X_i$'s. Argue that each $X_i$ is an independent
  random variable with a geometric distribution. 

\ifsolutions{\color{blue}{
\[ X = \sum\nolimits_{i=0}^n X_i \quad\quad\quad \text{Each} X_i \sim Geo(\frac{n-i}{n}) \]
}}\fi

\item We wish to use Chebyshev's inequality to bound the probability
  we have to buy substantially more than $n \ln{n}$ boxes. In order
  to do this, we need to compute the variance of a geometric random
  variable. There are multiple ways of doing this, including a
  recursive trick like that used by Lily's Lottery.

  Another approach is to use series techniques. We use the following
  lemma in our proof: 
  $\sum\nolimits_{k=1}^{\infty} k(k+1)(1-p)^{k-1} =
  \frac{2}{p^3}$. Prove this lemma. (Hint: what is the sum of the
  geometric series $\sum\nolimits_{k=0}^{\infty} (1-p)^k$. Take the
  derivative of both sides, what happens?)

\ifsolutions{\color{blue}{
\begin{align*}
\sum\nolimits_{k=0}^{\infty} (1-p)^k &= \frac{1}{p} &\text{take derivative of both sides twice} \\ 
-\sum\nolimits_{k=0}^{\infty} k(1-p)^{k-1} &= -\frac{1}{p^2}  &\text{1st}\\ 
\sum\nolimits_{k=0}^{\infty} k(k-1)(1-p)^{k-2} &= \frac{2}{p^3} &\text{2nd, LHS $k=0$ term is 0}\\ 
\sum\nolimits_{k=1}^{\infty} k(k-1)(1-p)^{k-2} &= \frac{2}{p^3} \\
\sum\nolimits_{j=0}^{\infty} (j+1)j(1-p)^{j-1} &= \frac{2}{p^3} 
&\text{change summation bounds} j = k-1 \\
\sum\nolimits_{j=1}^{\infty} j(j+1)(1-p)^{j-1} &= \frac{2}{p^3} &\text{LHS $j=0$ term is 0}
\end{align*}
}}\fi


\item If $X_i \sim Geom(p)$, show that $\mathbb{E}[X_i^2] = \sum\nolimits_{k=1}^{\infty} k(k+1)p(1-p)^{k-1} - \sum\nolimits_{k=1}^{\infty} kp(1-p)^{k-1}$

\ifsolutions{\color{blue}{
By definition $\mathbb{E}[X_i^2] = \sum\nolimits_{k=1}^{\infty} k^2p(1-p)^{k-1}$. Since $k^2 = k(k+1) - k$, we can plug that in to get the answer. 
}}\fi


\item Use your lemma and the fact that $\mathbb{E}[X_i] = 1/p$ to simplify part c) to show $\mathbb{E}[X_i^2] = \frac{2}{p^2} - \frac{1}{p} $

\ifsolutions{\color{blue}{
We simply plug in our lemma to the first sum term in part c). We also notice that the second sum term is just the expectation of a geometric variable with parameter $p$. So, we get 
\[ \mathbb{E}[X_i^2] = p\cdot \frac{2}{p^3} - \frac{1}{p} \]
}}\fi


\item Show that variance of a geometric variable with parameter $p$ is
  $\frac{1-p}{p^2}$. We will later use the simpler upper bound, $Var[X_i] < \frac{1}{p^2}$.

\ifsolutions{\color{blue}{
\begin{align*}
Var[X_i] &= \mathbb{E}[X_i^2] - \mathbb{E}[X_i]^2 \\ 
&= \frac{2}{p^2} - \frac{1}{p}  - (\frac{1}{p})^2 \\
&= \frac{1}{p^2} - \frac{1}{p} \\
&= \frac{1-p}{p^2} < \frac{1}{p^2}
\end{align*}
}}\fi


\item Make use of the fact that $\sum\nolimits_{i=1}^{\infty}
  \frac{1}{i^2}$ is a positive constant $\frac{\pi^2}{6} \leq 2$ to show that the $Var[X] \leq 2n^2$ 

\ifsolutions{\color{blue}{
We use the fact that the $X_i$'s are independent, 
\begin{align*}
Var[X] &= \sum\nolimits_{i=0}^{n-1} Var[X_i] \\ 
&\leq \sum\nolimits_{i=0}^{n-1} (\frac{n}{n-i})^2 \\
&= n^2\sum\nolimits_{i=0}^{n-1} \frac{1}{i}^2 \\
&\leq n^2\sum\nolimits_{i=0}^{\infty} \frac{1}{i}^2 \\
&\leq 2n^2
\end{align*}
}}\fi

\item This means that the standard deviation for $X$ scales like $n$
  and not like the expectation $n \ln n$. 

 Use Chebyshev's inequality to show that $Pr[ X \geq \alpha n
  \ln{n}]$ tends to zero for any $\alpha > 1$ as $n \rightarrow
  \infty$. (Hint: Recall that we estimated in the note that
  $\mathbb{E}[X] \approx n(\ln{n} + \gamma) \approx n \ln{n}$. ) 

\ifsolutions{\color{blue}{ FIX THESE
\begin{align*}
Pr[ X \geq n^{3/2} + n\ln{n}] &\approx Pr[ X \geq n^{3/2} + \mathbb{E}[X]] \\ 
&=  Pr[ X - \mathbb{E}[X] \geq n^{3/2} ] \\
&\leq Pr[ | X - \mathbb{E}[X] | \geq n^{3/2} ] \\
&\leq \frac{Var[X]}{n^3} \\
&\leq \frac{2n^2}{n^3} \\
&= \frac{2}{n}
\end{align*}
}}\fi


\end{enumerate}


\end{enumerate}

\end{document}
