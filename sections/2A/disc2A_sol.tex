\documentclass[11pt]{article}
\usepackage{../../svn_sty/cs70}
\usepackage{color}

\def\title{Discussion 2A}
\begin{document}
\maketitle

\begin{enumerate}

\item Use truth tables to show that $\neg (A\vee B) \,\equiv\, \neg A
\wedge \neg B$ and $\neg (A\wedge B) \,\equiv\, \neg A \vee \neg
B$. These two equivalences are known as DeMorgan's Law.


\begin{table}[h!]
\center
{\color{blue}{
\begin{tabular}{c|c||c|c||c|c}
$A$ & $B$ & $\lnot (A \lor B)$ & $ \lnot A \wedge \lnot B$ & $\lnot (A \wedge B)$ & $ \lnot A \lor \lnot B$ \\
\hline
0 & 0 & 1 & 1 & 1 & 1 \\
0 & 1 & 0 & 0 & 1 & 1 \\
1 & 0 & 0 & 0 & 1 & 1 \\
1 & 1 & 0 & 0 & 0 & 0 \\
\end{tabular}
}}
\end{table}

\item Which of the following statements are true? Let $Q(n)$
  be the statement ``$n$ is divisible by $2$.''  $\mathbb{N}$ denotes the
  set of natural numbers.
  \begin{enumerate}
    \item $\exists \ k\in \mathbb{N}, \ Q(k) \wedge Q(k+1)$.
    \item $\forall \ k\in \mathbb{N}, \ Q(k)\implies Q(k^2)$.
    \item $\exists \ x\in \mathbb{N}, \neg(\exists \ y\in \mathbb{N}, y < x)$.
  \end{enumerate}

{\color{blue}{(a) false, (b) true, (c) true}}

\item Write the following statements using the notation covered in class. Use $\mathbb{N}$ to denote the set of natural numbers and $\mathbb{Z}$ to denote the
set of integers.  Also write $P(n)$ for the statement ``$n$ is odd''.
\begin{enumerate}
\item For all natural numbers $n$, $2n$ is even. \\
{\color{blue}{$\forall \ n \in \mathbb{N}, \lnot P(2n)$}}
\item For all natural numbers $n$, $n$ is odd if $n^2$ is odd. \\
{\color{blue}{$\forall \ n \in \mathbb{N}, P(n^2) \implies P(n)$}}
\item There are no integer solutions to the equation $x^2 - y^2 = 10$. \\
{\color{blue}{$\lnot (\exists \ x,y \in \mathbb{Z}, x^2 - y^2 = 10)$}}
\end{enumerate}



\item You are on an island inhabited by two types of people: the Liars and the Truthtellers. Liars always make false statements, and Truthtellers always make true statements. In all other respects, the two types are indistinguishable. You meet an attractive local and ask him/her on a date. The local responds, ``I will go on a date with you if and only if I am a Truthteller.'' Is this good news?

{\color{blue}{Yes. Let $T$ be the proposition that the local is a Truthteller, and $D$ be the proposition that he/she will go on a date with you. If $T$ is true, then the statement $D \iff T$ is true, so we can find the value of $D$ in the last line of the truth table below.  It is a 1, so $D$ is true.  If $T$ is false, then the statement $T \iff D$ is false, so we can find the value of $D$ in the third line of the truth table below.  It is a 1 as well.}}

\begin{table}[h!]
\center
{\color{blue}{
\begin{tabular}{c|c|c}
T & D & $D \iff T$ \\
\hline
0 & 0 & 1 \\
0 & 1 & 0 \\
1 & 0 & 0 \\
1 & 1 & 1
\end{tabular}
}}
\end{table}

%\item 
%  Which of the following implications is/are true?
%  \begin{enumerate}
%    \item $\forall x\forall y \; P(x,y)$ implies $\forall y\forall x \; P(x,y)$.
%    \item $\exists x\exists y \; P(x,y)$ implies $\exists y\exists x \; P(x,y)$.
%    \item $\forall x\exists y \; P(x,y)$ implies $\exists y\forall x \; P(x,y)$.
%    \item $\exists x\forall y \; P(x,y)$ implies $\forall y\exists x \; P(x,y)$.
%  \end{enumerate}
%  Also, for the implication in part (c), what is its converse? And its
%  contrapositive?
%
%\item Prove or disprove each of the following:
%\begin{enumerate}
%\item $\forall x \bigl( P(x) \wedge Q(x) \bigr)~\equiv~\forall x P(x) \wedge \forall x Q(x)$
%\item $\forall x \bigl( P(x) \vee Q(x) \bigr)~\equiv~\forall x P(x) \vee \forall x Q(x)$
%\item $\forall x \bigl( P(x) \Rightarrow Q(x) \bigr)~\equiv~\bigl( \forall x P(x) \bigr) \Rightarrow \bigl( \forall x Q(x) \bigr)$
%\item $\exists x \bigl( P(x) \wedge Q(x) \bigr)~\equiv~\exists x P(x) \wedge \exists x Q(x)$
%\item $\exists x \bigl( P(x) \vee Q(x) \bigr)~\equiv~\exists x P(x) \vee \exists x Q(x)$
%\item $\exists x \bigl( P(x) \Rightarrow Q(x) \bigr)~\equiv~\bigl( \exists x P(x) \bigr) \Rightarrow \bigl( \exists x Q(x) \bigr)$
%\end{enumerate}
%
%\item Complete the following expression so that it states that: ``There is one
%  and only one natural number $n$ for which the proposition formula $P(n)$
%  holds.''
%  \[ (\exists n\in \N) \dots \]

%\item A valid tiling for a chessboard is an arrangement of tiles such that no two tiles overlap and every square of the board is covered by a tile.
%\begin{enumerate}
%\item A domino is a tile consisting of two contiguous squares. Is there a valid domino tiling for the $8 \times 8$ chessboard where the squares in the bottom left and top right corners have been removed?
%\item A straight tetromino is a tile consisting of four contiguous squares. Prove or disprove: A $10 \times 10$ chessboard can be tiled with straight tetrominoes.
%\end{enumerate}

\item Prove that if you put $n+1$ apples into $n$ boxes, any way you like, then at least one box must contain 2 apples.  This is known as the {\em pigeon hole principle}. \\

{\color{blue}{Suppose this is not the case.  Then all the boxes would contain at most 1 apple. Then the maximum number of apples we could have would be $n$, but this is a contradiction since we have $n+1$ apples.}}

\item Prove that the length of the hypotenuse of a non-degenerate right triangle is strictly greater than the sum of the two remaining sides.
\begin{enumerate}
\item Write down the definition of a right triangle and the claim to be proven in mathematical notation.
\item Prove the statement by contradiction.
\item Prove the statement directly.
\end{enumerate}

{\color{blue}{Definition of a right triangle: $a^2 + b^2 = c^2$.  Claim to be proven: $a+b > c$. We can prove this directly by adding $2ab$ (a positive number) to the LHS of the definition of a right triangle.
\[ a^2 + b^2 = c^2 \implies a^2 + 2ab + b^2 > c^2 \implies (a+b)^2 > c^2 \implies a+b > c \]
We can prove the claim with contradiction by assuming it is not true. This is basically the reverse of the previous proof:
\[ a + b \leq c \implies (a+b)^2 \leq c^2 \implies a^2 + 2ab + b^2 \leq c^2 \implies a^2 +b^2 < c^2 \]
\[ \implies \Longleftarrow \]
}}

\end{enumerate}
\end{document}
