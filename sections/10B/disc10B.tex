\documentclass[11pt]{article}
\usepackage{../../cs70}
\usepackage{color}
\newif\ifsolutions
\solutionstrue
\solutionsfalse %flag for solutions

\def\title{Discussion 10B}
\begin{document}
\maketitle


\begin{enumerate}


\item {\bf Prove!} 

  Consider the following identity: $$ \binom{n}{k} = \binom{n-1}{k-1} + \binom{n-1}{k} $$

  \begin{enumerate}
  \item Prove the identity by algebraic manipulation.

	\vspace{30mm}

    \ifsolutions{\color{blue}
	\vspace{-30mm}
      {\begin{eqnarray*}
          \binom{n}{k} &=& \frac{n!} {(n-k)!\cdot k!} \\
          &=& \frac{n}{k}\cdot \frac{(n-1)!} {(n-k)!\cdot (k-1)!} \\
          &=& \frac{n+k-k}{k}\cdot \frac{(n-1)!} {(n-k)!\cdot (k-1)!} \\
          &=& \frac{k}{k}\cdot\frac{(n-1)!} {(n-k)!\cdot (k-1)!} + \frac{n-k}{k}\cdot \frac{(n-1)!} {(n-k)!\cdot (k-1)!}\\
          &=& \frac{(n-1)!} {(n-k)!\cdot (k-1)!} + \frac{(n-1)!} {(n-k-1)!\cdot k!}\\
          &=& \binom{n-1}{k-1} + \binom{n-1}{k}
          \end{eqnarray*}
    }} \fi
    
  \item Prove the identity using a combinatorial argument.  

\vspace{30mm}
    
    \ifsolutions{\color{blue}
\vspace{-30mm}
      {The left hand side is the number of ways to choose $k$ elements    
        out of $n$. 

        Looking at this another way, we look at the first element and decide whether 
        we are going to choose it or not. If we choose it, then we need to choose $k-1$
        more elements from the remaining $n-1$. If we don't choose it, then we need
        to choose all our $k$ elements from the remaining $n-1$.
        We are not double counting any as in one of our cases, we chose the first
        element and in the other, we did not.
        We are counting not leaving everything out by separating it into these cases.

        This completes the proof.
      }} \fi
  \end{enumerate}


  Consider the following identity: $$ \binom{2n}{2} = 2 \binom{n}{2} + n^2. $$
  \begin{enumerate}
  \item Prove the identity by algebraic manipulation.
	\vspace{30mm}

    \ifsolutions{\color{blue}
	\vspace{-30mm}
      {\begin{eqnarray*}
          \binom{2n}{2} &=& \frac{2n(2n-1)} {2} \\
          &=& n(2n-1) \\
          &=& n(n-1) + n^2 \\
          &=& 2\frac{n(n-1)}{2} + n^2 \\
          &=& 2\binom{n}{2} + n^2.
          \end{eqnarray*}
    }} \fi
    
  \item Prove the identity using a combinatorial argument.  
    
    \ifsolutions{\color{blue}
      {The left hand side is the number of ways to choose two elements    
        out of $2n$. Counting in another way, we first divide the $2n$
        elements (arbitrarily) into two sets of $n$ elements.  Then we consider
        three cases: either we choose both
        elements out of the first $n$-element set, both out of the second 
        $n$-element set, or one element out of each set. The number of ways we
        can do each of these things is $\binom{n}{2}$, $\binom{n}{2}$, and $n^2$,
        respectively. Since these three cases are mutually exclusive and cover
        all the  possibilities, summing them must give the same number 
        as the left hand side.  This completes the proof.
      }} \fi
  \end{enumerate}

  
\newpage
\item {\bf Permutations }
  
  Consider $n$ people. Each person has a role model among the $n$ people (which could be the person herself/himself) and it turns out that
  all role models are unique. In other words the role models form a bijection from the set of $n$ people to themselves, which
  is also called a permutation. In this permutation, we can consider cycles. If we start from a person $A$, then her role model
  is going to be a person $B$ who also has a role model $C$ and so on.
  \begin{itemize}
    \item Prove that if we continue following the chain of role models this way, we have to return to $A$ before we meet any other repetition.
	\vspace{20mm}

      \ifsolutions{\color{blue}
	\vspace{-20mm}
        We must see a repetition at some point. But every person other than $A$ has been seen as a role model of someone else (i.e. the previous person).
        Since role models are unique, no other person can be the first repetition (otherwise that person will be the role model of two different people).
      } \fi
    \item The length of a cycle is an interesting number. Fix a person $A$ and a number $k$. How many different permutations result in that person being in 
      a cycle of length $k$? If all permutations were equality likely, what is the corresponding probability (i.e. the count divided by the total number
      of permutations)?

	\vspace{20mm}
      \ifsolutions{\color{blue}
	\vspace{-20mm}
        First we must select the $k$ members of the cycle. This can be done in ${n-1 choose k-1}$ ways. Then we must permute them in order to form a cycle (
        we keep $A$ as the head of the cycle to avoid repetition). This can be done in $(k-1)!$ ways. Then we construct a permutation on the remaining $n-k$
        people in $(n-k)!$ ways. So the total number is $(n-k)!(k-1)!{n-1\choose k-1}=(n-1)!$. This translates to a probability of $1/n$.
      } \fi
  \end{itemize}



%\newpage

\item {\bf Balls and bins}
  
  You have $n$ bins and you throw balls into them one by one randomly.
  A collision is when a ball is thrown into a bin which already has another ball.
  \begin{enumerate}
  \item What is the probability that the first ball thrown will cause the first collision?
	\vspace{5mm}
  \item What is the probability that the second ball thrown will cause the first collision?
	\vspace{5mm}
  \item What is the probability that, given the first two balls are not in collision, 
    the third ball thrown will cause the first collision?
	\vspace{10mm}
  \item What is the probability that the third ball thrown will cause the first collision? 
	\vspace{10mm}
  \item What is the probability that, given the first $m-1$ balls are not in collision, 
    the $m^{th}$ ball thrown will cause the first collision?
	\vspace{10mm}
  \item What is the probability that the $m^{th}$ ball thrown will cause the first collision? 
  \end{enumerate}

\ifsolutions
{\color{blue}{
  {\bf Solutions:}
  \begin{enumerate}
  \item 0
  \item $\frac{1}{n}$
  \item $\frac{2}{n}$
  \item Basically: $P(\text{Ball 3 collides $|$ Ball 1, 2 do not collide}) \cdot P(\text{Ball 1, 2 do not collide})$

    Which is $\frac{2}{n} \cdot \frac{n-1}{n}$
  \item $\frac{m-1}{n}$
  \item Similar to (d), $\frac{m-1}{n} \cdot \frac{n-1}{n} \cdot \frac{n-2}{n} \cdot \hdots \cdot \frac{n-m+2}{n} = 
    \frac{m-1}{n} \cdot \prod\limits_{i=0}^{m-2} \frac{n-i}{n}$
  \end{enumerate}
}}
\fi

%\qns{Another Stirling Approximation}
%
%\begin{enumerate}
%\item Consider a deck with $M$ many distinct cards. Suppose one removes a card from this deck at random $x$ many times, recording the ordered sequence of drawn cards. How many possible sequences could one end up with?
%\item Suppose instead that, after each draw, the drawn card is reinserted into the deck. Now how many possible sequences could one get (when drawing randomly with replacement $x$ many times from a deck of $M$ cards)?
%\item What is the probability that no card gets drawn more than once, if one draws randomly with replacement $x$ many times from a deck of $M$ cards?
%\item What happens to this probability as $x$ is held constant but $M$ grows very large?
%\item Show that, for any natural numbers $x$ and $N$, we have that $\displaystyle x! = \frac{(N + x)!}{N!} \prod_{i = 1}^{N} \frac{i}{x + i}$.
%\item Combining the previous parts, show that $\displaystyle x! = \lim_{N \rightarrow \infty} (N + x)^x \prod_{i = 1}^{N} \frac{i}{x + i}$. (\textit{Hint: Think of a deck with $N + x$ many cards.})
%\end{enumerate}
%Note that this last formula can be made sense of even for non-integer $x$; this is often used to generalize the factorial function in mathematics.
%
%
%\ifsolutions
%{\color{blue}{
%  {\bf Solutions:}
%\begin{enumerate}
%\item $M!/(M - x)!$.
%\item $M^x$.
%\item $\frac{M!/(M - x)!}{M^x} = \displaystyle \prod_{i = 0}^{x - 1} \frac{M - i}{M} = \prod_{i = 0}^{x - 1} (1 - i/M)$.
%\item As seen in at the end of the last answer sketch, this is the product of a constant number of factors, each of which approaches $1$ as $M$ grows very large. Thus, this probability itself approaches $1$ as $M$ grows very large.
%\item Simply note that the product on the right is $\frac{N!}{(N + x)!/x!}$, and then cancel out the $N!$ and $(N + x)!$.
%\item Take the result from the previous section, and subsitute $(N + x)^x$ for $\frac{(N + x)!}{N!}$, using the fact that their limiting ratio is $1$ for large $N$ (as established two sections ago).
%\end{enumerate}
%}}
%\fi

   
\end{enumerate}

\end{document}
