\documentclass[11pt]{article}
\usepackage{../../cs70}
\usepackage{color}
\newif\ifsolutions
\solutionstrue
\solutionsfalse %flag for solutions

\def\title{Discussion 10B}
\begin{document}
\maketitle


\begin{enumerate}

\item {\bf Balls and bins}
  
  You have $n$ bins and you throw balls into them one by one randomly.
  A collision is when a ball is thrown into a bin which already has another ball.
  \begin{enumerate}
  \item What is the probability that the first ball thrown will cause the first collision?
  \item What is the probability that the second ball thrown will cause the first collision?
  \item What is the probability that, given the first two balls are not in collision, 
    the third ball thrown will cause the first collision?
  \item What is the probability that the third ball thrown will cause the first collision? 
  \item What is the probability that, given the first $m-1$ balls are not in collision, 
    the $m^{th}$ ball thrown will cause the first collision?
  \item What is the probability that the $m^{th}$ ball thrown will cause the first collision? 
  \end{enumerate}

\ifsolutions
{\color{blue}{
  {\bf Solutions:}
  \begin{enumerate}
  \item 0
  \item $\frac{1}{n}$
  \item $\frac{2}{n}$
  \item Basically: $P(\text{Ball 3 collides $|$ Ball 1, 2 do not collide}) \cdot P(\text{Ball 1, 2 do not collide})$

    Which is $\frac{2}{n} \cdot \frac{n-1}{n}$
  \item $\frac{m-1}{n}$
  \item Similar to (d), $\frac{m-1}{n} \cdot \frac{n-1}{n} \cdot \frac{n-2}{n} \cdot \hdots \cdot \frac{n-m+2}{n} = 
    \frac{m-1}{n} \cdot \prod\limits_{i=0}^{m-2} \frac{n-i}{n}$
  \end{enumerate}
}}
\fi

\qns{Another Stirling Approximation}

\begin{enumerate}
\item Consider a deck with $M$ many distinct cards. Suppose one removes a card from this deck at random $x$ many times, recording the ordered sequence of drawn cards. How many possible sequences could one end up with?
\item Suppose instead that, after each draw, the drawn card is reinserted into the deck. Now how many possible sequences could one get (when drawing randomly with replacement $x$ many times from a deck of $M$ cards)?
\item What is the probability that no card gets drawn more than once, if one draws randomly with replacement $x$ many times from a deck of $M$ cards?
\item What happens to this probability as $x$ is held constant but $M$ grows very large?
\item Show that, for any natural numbers $x$ and $N$, we have that $\displaystyle x! = \frac{(N + x)!}{N!} \prod_{i = 1}^{N} \frac{i}{x + i}$.
\item Combining the previous parts, show that $\displaystyle x! = \lim_{N \rightarrow \infty} (N + x)^x \prod_{i = 1}^{N} \frac{i}{x + i}$. (\textit{Hint: Think of a deck with $N + x$ many cards.})
\end{enumerate}
Note that this last formula can be made sense of even for non-integer $x$; this is often used to generalize the factorial function in mathematics.


\ifsolutions
{\color{blue}{
  {\bf Solutions:}
\begin{enumerate}
\item $M!/(M - x)!$.
\item $M^x$.
\item $\frac{M!/(M - x)!}{M^x} = \displaystyle \prod_{i = 0}^{x - 1} \frac{M - i}{M} = \prod_{i = 0}^{x - 1} (1 - i/M)$.
\item As seen in at the end of the last answer sketch, this is the product of a constant number of factors, each of which approaches $1$ as $M$ grows very large. Thus, this probability itself approaches $1$ as $M$ grows very large.
\item Simply note that the product on the right is $\frac{N!}{(N + x)!/x!}$, and then cancel out the $N!$ and $(N + x)!$.
\item Take the result from the previous section, and subsitute $(N + x)^x$ for $\frac{(N + x)!}{N!}$, using the fact that their limiting ratio is $1$ for large $N$ (as established two sections ago).
\end{enumerate}
}}
\fi

   
\end{enumerate}

\end{document}
